\subsection{Berry Curvature}

\subsubsection{Normal States}

For all normal eigenstates, the band resolved Berry curvature is
\begin{equation}
  \vc{Ω}_{τ σ}^n \ofK
  = i ∇_{\vK} ⨯
    \Braket{u_{τ σ}^n \ofK | ∇_{\vK} | u_{τ σ}^n \ofK}.
\end{equation}

This can be computed directly by first considering
\begin{multline}
  \Braket{u_{τ σ}^n \ofK | ∇_{\vK} | u_{τ σ}^n \ofK} \\
  = ∑_ν
    \cc{M}_{τ σ}^{ν n} \ofK
    ∇_{\vK} M_{τ σ}^{ν n} \ofK \\
  + ∑_{ν, ν'}
    \cc{M}_{τ σ}^{ν n} \ofK
    M_{τ σ}^{ν' n} \ofK
    \Braket{v_{τ σ}^ν \ofK | ∇_{\vK} | v_{τ σ}^{ν'} \ofK}.
\end{multline}
The second term is effectively zero by an argument
similar to the one given for the optical approximation above.
Using the identities
\begin{subequations}
  \begin{align}
    \pderiv{}{k} M_{τ σ}^{ν n} \ofK
    & = n τ \pderiv{}{k} \fnTheta{n}, \\
    \pderiv{}{ϕ} M_{τ σ}^{ν n} \ofK
    & = i τ M_{τ σ}^{ν n} \ofK δ_{ν, -1},
  \end{align}
\end{subequations}
gives the $z$-component of the curvature,
\begin{subequations}
  \begin{align}
    Ω_{τ σ}^n \of{k}
    & = \vc{\hat{z}} · \vc{Ω}_{τ σ}^n \ofK \\
    & = - n τ
      \left[ \frac{1}{2 k} \pderiv{}{k} \fnTheta{n} \right]
      \sin{\fnTheta{n}}, \\
    & = - n τ
      \frac{2 {(a t)}^2 (Δ - λ τ σ)}
      {{\left[{(2 a t k)}^2 + {(Δ - λ τ σ)}^2 \right]}^{3/2}}.
  \end{align}
\end{subequations}

The Berry curvature of left and right circularly polarized
($\vc{ϵ_±}$) optical excitations for a given $\vK$
then follows to be $± 2 Ω_{+ ↑}^+ \of{k}$.

\subsubsection{BCS States}

For simplicity,
we consider the case where $γ = - γ' = +1$
and $Δ_{\vK}$ is real in the BCS Hamiltonian in
\cref{eq:hamiltonian:bcs:general}.
The BCS ground states is
\begin{subequations}
  \begin{align}
    \Ket{Ω}
    & = ∏_{\vK} \csc{β_{\vK}} b_{\vK ↑} b_{-\vK ↓} \Ket{0}, \\
    & = ∏_{\vK} \left( \cos{β_{\vK}} - \sin{β_{\vK}}
        c_{\vK ↑}^† c_{-\vK ↓}^† \right) \Ket{0}.
  \end{align}
\end{subequations}%
\footnote{%
  Note that the full ground state
  also contains the lower two filled bands,
  but those contribute zero net Berry curvature and may be ignored
  in this section and the next.}
This BCS ground state may be viewed as built up
from the single-quasiparticle eigenstates,
\begin{equation}
  \Ket{\vK}
  = \csc{β_{\vK}} b_{\vK ↑} b_{-\vK ↓} \Ket{0},
\end{equation}
of the $\vK$ dependent Hamiltonian
$λ_{\vK} b_{\vK ↑}^† b_{\vK ↑}
- λ_{\vK} b_{-\vK ↓} b_{-\vK ↓}^†$.
Thus, consider the $z$-component of the Berry curvature of this state,
\begin{subequations}
  \begin{align}
    Ω_{\vK}
    & = i ∇_{\vK} ⨯
    \Braket{\vK | ∇_{\vK} | \vK}, \\
    & = i ∇_{\vK} ⨯
    \Braket{0 | c_{-\vK ↓} c_{\vK ↑}
      ∇_{\vK} \left( c_{\vK ↑}^† c_{-\vK ↓}^† \right) |0}, \\
    & = Ω_{+ ↑}^- \of{k} + Ω_{- ↓}^- \of{-k} = 0.
  \end{align}
\end{subequations}

\subsubsection{Berry Curvature from Optical Excitation}

A single optically excited state in the left valley
for a given $\vK$ is
\begin{subequations}
  \begin{align}
    & {c_{+ ↑}^+}^† \ofK c_{\vK ↑} \Ket{\vK}, \\
    & = {c_{+ ↑}^+}^† \ofK
      \left( \cos{β_{\vK}} b_{\vK ↑} + \sin{β_{\vK}} b_{-\vK ↓}^† \right)
      \Ket{\vK}, \\
    & = \sin{β_{\vK}} {c_{+ ↑}^+}^† \ofK b_{-\vK ↓}^† \Ket{\vK}, \\
    & = {c_{+ ↑}^+}^† \ofK b_{-\vK ↓}^† b_{\vK ↑} b_{-\vK ↓} \Ket{0}, \\
    & = - \sin^2 {β_{\vK}} {c_{+ ↑}^+}^† \ofK b_{\vK ↑} \Ket{0}, \\
    & = - \sin^3 {β_{\vK}} {c_{+ ↑}^+}^† \ofK {c_{- ↓}^-}^† \ofMK \Ket{0},
  \end{align}
\end{subequations}
which has corresponding Berry Curvature
\begin{subequations}
  \begin{align}
    Ω_{\vK}^L
    & = \sin^6 {β_{\vK}}
        \left[ Ω_{+ ↑}^+ \of{k} + Ω_{- ↓}^- \of{-k} \right], \\
    & = 2 \sin^6 {β_{\vK}} Ω_{+ ↑}^+ \of{k}.
  \end{align}
\end{subequations}
A single optically excited state in the right valley
for a given $\vK$ is
\begin{subequations}
  \begin{align}
    & {c_{- ↓}^+}^† \ofMK c_{-\vK ↓} \Ket{\vK}, \\
    & = {c_{- ↓}^+}^† \ofK
      \left( - \cos{β_{\vK}} b_{-\vK ↓} + \sin{β_{\vK}} b_{\vK ↑}^† \right)
      \Ket{\vK}, \\
    & = \sin{β_{\vK}} {c_{- ↓}^+}^† \ofK b_{\vK ↑}^† \Ket{\vK}, \\
    & = {c_{- ↓}^+}^† \ofK b_{\vK ↑}^† b_{\vK ↑} b_{-\vK ↓} \Ket{0}, \\
    & = \sin^2 {β_{\vK}} {c_{- ↓}^+}^† \ofK b_{-\vK ↓}\Ket{0}, \\
    & = \sin^3 {β_{\vK}} {c_{- ↓}^+}^† \ofK {c_{+ ↑}^-}^† \ofK \Ket{0},
  \end{align}
\end{subequations}
which has corresponding Berry Curvature
\begin{subequations}
  \begin{align}
    Ω_{\vK}^R
    & = \sin^6 {β_{\vK}}
        \left[ Ω_{- ↓}^+ \of{k} + Ω_{+ ↑}^- \of{k} \right], \\
    & = - 2 \sin^6 {β_{\vK}} Ω_{+ ↑}^+ \of{k}, \\
    & = - Ω_{\vK}^L.
  \end{align}
\end{subequations}
