\subsection{Induced Superconductivity}

A proximity $s$-wave superconductor will inject cooper pairs
into the atomic orbitals according to
\begin{equation}
  H^V
  = \sumK ∑_{ν, τ} \cc{Δ}_ν
    a^ν_{-τ ↓} \ofMK a^ν_{τ ↑} \ofK + \frac{ε}{2} + \hc
\end{equation}
The coupling constants $Δ_ν$ and the overall constant $ε$
depend on the material interface.

The Hamiltonian projected to the upper $n = -1$ bands with $τ = σ$ is
\begin{multline}
  \label{eq:induced:projected}
  P_{τ = σ}^{-} \left( H^0 + H^V - μ N \right)
  = \sumK ∑_α ξ_{\vK} c_{\vK α}^† c_{\vK α} \\
    - \sumK \left( \cc{Δ}_{\vK} c_{-\vK ↓} c_{\vK ↑}
    + Δ_{\vK} c_{\vK ↑}^† c_{-\vK ↓}^† \right)
    + ε,
\end{multline}
where
\begin{equation}
  Δ_{\vK}
  = \frac{1}{2} \left( Δ_{\text{c}} + Δ_{\text{v}} \right)
    +
    \frac{1}{2} \left( Δ_{\text{c}} - Δ_{\text{v}} \right)
    \cos{θ_{\vK}}.
\end{equation}
We use an abbreviated notation where we suppress the band index
and write $α = ↑,↓$ for $τ = σ = ±1$.
With this, let $ξ_{\vK} = E_{+ ↑}^- \ofAbsK - μ$
and $c_{\vK α} = c^-_α \ofK$,
where $μ$ is the chemical potential.
The prime on the sum over $\vK$ indicates the summation is restricted
to a suitable region within each valley.

\Cref{eq:induced:projected}
is identical in form to the standard BCS Hamiltonian.
We may take $Δ_{\vK}$ to be real
(otherwise, if $Δ_{\vK} = \abs{Δ_{\vK}} e^{i \arg{Δ_{\vK}}}$,
then make the unitary transformation
$c_{\vK α} → e^{i \arg{Δ_{\vK}} / 2} c_{\vK α}$).
The diagonalized form is
\begin{multline}
  P_{τ = σ}^{-} \left( H^0 + H^V - μ N \right) = \\
    \sumK ∑_α λ_{\vK} b_{\vK α}^† b_{\vK α}
    + \sumK \left( ξ_{\vK} -  λ_{\vK} \right) + ε,
\end{multline}
with eigenvalues
\begin{equation}
  λ_{\vK} = \sqrt{ξ_{\vK}^2 + Δ_{\vK}^2}.
\end{equation}
This follows from the standard Bogoliubov transformation,
written compactly as
\begin{equation}
  c_{\vK α}
  = α \cos{β_{\vK}} b_{\vK α} + \sin{β_{\vK}} b_{-\vK, -α}^†,
\end{equation}
where
\begin{subequations}
  \begin{alignat}{3}
    \sin{2 β_{\vK}} & = && {}-{} && Δ_{\vK} / λ_{\vK}, \\
    \cos{2 β_{\vK}} & = && {} {} && ξ_{\vK} / λ_{\vK}.
  \end{alignat}
\end{subequations}
Note also the inverse,
\begin{equation}
  b_{\vK α}
  = α \cos{β_{\vK}} c_{\vK α} + \sin{β_{\vK}} c_{-\vK, -α}^†,
\end{equation}
