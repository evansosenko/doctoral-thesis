\label{s:appendix:optical}
\section{Single electron transitions}

Consider the spin-orbit Hamiltonian for a single non-interacting electron
which includes a position dependent potential $W \of{\vc{Q}}$
and an electromagnetic potential $\vc{A} \of{\vc{Q}}$,
\begin{equation}
  H
  = \frac{{\left( \vc{P} + e \vc{A} \right)}^2}{2 M}
  + λ \vc{L} · \vc{S} + W.
\end{equation}
Given $\vc{L} = \vc{Q} × \vc{P}$, one can show, either via
the commutation relations or formal differentiation,
that the velocity operator is
\begin{equation}
  \vc{V}
  = i \left[ H, \vc{Q} \right]
  = \pderiv{H}{\vc{P}}
  = \frac{\vc{P} + e \vc{A}}{M} + λ \vc{Q} × \vc{S}.
\end{equation}
In a gauge with $∇ ·\vc{A} = 0$,
\begin{equation}
  H = \frac{\vc{P}^2}{2 M} + \frac{e}{M} \vc{A} · \vc{P} + e^2 \vc{A}^2.
\end{equation}

For monochromatic optical perturbations
with amplitude $A_0$,
wave vector $\vc{q}$,
frequency $ω$,
and polarization vector $\vc{ϵ}$,
the electromagnetic potential is of the form
\begin{equation}
  \vc{A} = 2 \re{\vc{ϵ} A_0 e^{i \left( \vc{q} · \vc{Q} - ω t \right)}}.
\end{equation}
For linear perturbations, we neglect the term of order $\vc{A}^2$
to write the Hamiltonian in the form
\begin{equation}
  H = H_0 + H' e^{- i ω t} + {H'}^† e^{i ω t}.
\end{equation}
Here,
$H' = e A_0 e^{i \vc{q} · \vc{Q}} \left( \vc{ϵ} · \vc{V} \right)$
appears as a standard harmonic perturbation where
$\vc{V} = i [ H_0, \vc{Q} ]$
is the velocity operator for the unperturbed system.

Thus, according to Fermi's golden rule,
the optical transition probability per unit time is
\begin{equation}
  Γ_{f ← i}
  = 2 π e^2 A_0^2
    \abs{\Braket{f | e^{i \vc{q} · \vc{Q}} \vc{ϵ} · \vc{V} | i}}^2
    δ \of{E_f + E_i - ω}. % chktex 19
\end{equation}
Typically, the dipole approximation is used in which
$e^{i \vc{q} · \vc{Q}} → 1$.

\subsection{Tight binding $d$-orbital transitions}
\label{s:appendix:optical:d-orbital}

In a noninteracting system of Bloch electrons,
\begin{equation}
  ∇_{\vK} H_{\vK} = \vc{V}_{\vK}.
\end{equation}
Thus, for each $\vK$,
the matrix element appearing in the optical transition rate
(in the dipole approximation) may be computed as
\begin{equation}
  \Braket{ψ_{n' \vK} | \vc{V} | ψ_{n \vK}}
  = \Braket{u_{n' \vK} | \vc{V}_{\vK} | u_{n \vK}}
  = \Braket{u_{n' \vK} | ∇_{\vK} H_{\vK} | u_{n \vK}}.
\end{equation}

The matrix elements of the derivative,
$\Braket{u_{n \vK} | ∇_{\vK} H_{\vK} | u_{n' \vK}}$,
appearing in the optical translation rate may be cumbersome to compute.
We show that in the tight binding approximation with only $d$-type orbitals,
it is sufficient to compute the derivative of the orbital matrix elements,
$∇_{\vK} H_{\vK}^{ν ν'}
= ∇_{\vK} \Braket{v_{ν \vK} | H_{\vK} | v_{ν' \vK}}$.

We have
\begin{equation}
  H_{\vK}
     = ∑_{n \vK'} E_{n \vK'} T_{\vK} \Ket{ψ_{n \vK'}}
        \Bra{ψ_{n \vK'}} T_{\vK}^{-1}
     = H_{\vK}^0 + H_{\vK}^1,
\end{equation}
where
\begin{align}
  \label{appendix:optical:decomposition}
  H_{\vK}^0
  & = ∑_{ν, ν'}
      H_{\vK}^{ν ν'}
      \Ket{v_{ν \vK}} \Bra{v_{ν' \vK}}, \\
  H_{\vK}^1
  & = ∑_{ν, ν'} ∑_{\vK' ≠ \vK}
      H_{\vK'}^{ν ν'}
      T_{\vK - \vK'}
      \Ket{v_{ν \vK'}} \Bra{v_{ν' \vK'}}
      T_{\vK - \vK'}^{-1}.
\end{align}
First, using
$∇_{\vK} H_{\vK}^1 = i \left[ H_{\vK}^1, \vc{Q} \right]$,
we obtain a sum over $\vK' ≠ \vK$ of terms proportional
to $δ_{\vK, \vK'}$, % chktex 19
thus all matrix elements for $∇_{\vK} H_{\vK}^1$ vanish.
Next, we have
\begin{subequations}
  \begin{align}
    \label{appendix:optical:sum.1}
    ∇_{\vK} H_{\vK}^0
    & = ∑_{ν, ν'}
      ∇_{\vK} H_{\vK}^{ν ν'}
      \Ket{v_{ν \vK}} \Bra{v_{ν' \vK}} \\
    \label{appendix:optical:sum.2}
    & + ∑_{ν, ν'}
      H_{\vK}^{ν ν'}
      \left( ∇_{\vK} \Ket{v_{ν \vK}} \right) \Bra{v_{ν' \vK}} \\
    \label{appendix:optical:sum.3}
    & + ∑_{ν, ν'}
      H_{\vK}^{ν ν'}
      \Ket{v_{ν \vK}} \left( ∇_{\vK}  \Bra{v_{ν' \vK}} \right).
  \end{align}
\end{subequations}
Using
\begin{equation}
  \label{appendix:optical:ket:del}
  ∇_{\vK} \Ket{v_{ν \vK}}
  = \frac{T_{\vK}}{i \sqrt{N}}
    ∑_{n = 1}^N e^{i \vK ⋅ \vRn{n}}
    T \of{\vRn{n}} \vc{Q} \Ket{φ_ν},
\end{equation}
we find the sum
\cref{appendix:optical:sum.2}
contains terms proportional to the local optical matrix elements
$\Braket{φ_{ν'} | T \of{\vRn{n}} \vc{Q} | φ_ν}$.
For finite $\vRn{n}$, these off-center integrals are small,
and for $\vRn{n} = 0$, optical transitions between $d$-orbitals
are forbidden by symmetry.
Similar logic applies to the sum
\cref{appendix:optical:sum.1},
thus
\begin{equation}
  \label{appendix:optical:approx}
  \Braket{u_{n \vK} | ∇_{\vK} H_{\vK} | u_{n' \vK}}
  ⋍ ∑_{ν, ν'}
    \cc{M}_{\vK}^{ν n}
    \left( ∇_{\vK} H_{\vK}^{ν ν'} \right)
    M_{\vK}^{ν' n'}.
\end{equation}
In the case where $H_{\vK}^{ν ν'}$ is linear in $\vK$,
computing this derivative is equivalent to using minimal substitution, i.e.,
$H_{\vK + e \vc{A}}^{ν ν'} = H_{\vK}^{ν ν'} + e A ∇_{\vK} H_{\vK}^{ν ν'}$.

\section{Transitions in the TMD model}

As remarked above, we may consider a perturbation of
$H_{τ σ}^{ν ν'} \ofK$ arising from minimal coupling,
\begin{equation}
  H_{τ σ}^{ν ν'} \ofK
  → H_{τ σ}^{ν ν'} \of{\vK + e \vc{A}}
  = H_{τ σ}^{ν ν'} \ofK + h_τ^{ν ν'},
\end{equation}
where in the dipole approximation,
$\vc{A} = 2 \re{\vc{ϵ} A_0 e^{- i ω t}}$,
$h_τ^{v v} = h_τ^{c c} = 0$,
and
\begin{equation}
  h_τ^{v c}
  = \cc{h}_τ^{c v}
  = 2 a t e A_0
    \left( τ \vc{\hat{x}} + i \vc{\hat{y}} \right)
    · \re{\left( \vc{ϵ} e^{- i ω t} \right)}.
\end{equation}
Thus the optical perturbation operator is
\begin{equation}
  H^A
  = \sumK ∑_{τ, σ}
    h_τ^{v c}
    {a^v_{τ σ}}^† \ofK
    a^c_{τ σ} \ofK
    + \hc
\end{equation}
Separating out the time dependence,
we may write this in the standard form
$H^A = H' e^{- i ω t} + H'^† e^{i ω t}$,
where
\begin{equation}
  H'
  = \sumK ∑_{τ, σ}
    H'_τ
    {a^v_{τ σ}}^† \ofK
    a^c_{τ σ} \ofK
  - \sumK ∑_{τ, σ}
    H'_{-τ}
    {a^c_{τ σ}}^† \ofK
    a^v_{τ σ} \ofK,
\end{equation}
and
\begin{equation}
  H'_τ
  = a t e A_0
    \left( τ \vc{\hat{x}} + i \vc{\hat{y}} \right)
    · \vc{ϵ}.
\end{equation}
Chancing basis, we recover the optical matrix element,
$P_{τ σ}^{n n'} \of{\vK, \vc{ϵ}}$,
now explicitly a function of the polarization vector,
\begin{equation}
  H^A
  = \sumK ∑_{τ, σ} ∑_{n, n'}
    \frac{e A_0}{m_0}
    P_{τ σ}^{n n'} \of{\vK, \vc{ϵ}}
    {c_{τ σ}^n}^† \ofK
    c_{τ σ}^{n'} \ofK.
\end{equation}
For circularly polarized light,
$\vc{ϵ_±} = \left( \vc{\hat{x}} ± i \vc{\hat{y}} \right) / \sqrt{2}$
and
\begin{equation}
  P_{τ σ}^{+ -} \of{\vK, \vc{ϵ_±}}
  = ∓ τ \sqrt{2} a t m_0
    e^{± i ϕ}
    \sin^2 {\frac{\fnTheta{∓ τ}}{2}}.
\end{equation}

A brief note about the units of $P$ above.
To conveniently express $P$ in units of energy,
multiply by $c / \si{\planckbar}$.
Typically, $a t$ is given in \si{\angstrom\electronvolt},
the electron mass is given in units
of $\si{\mega \electronvolt} / c^2$,
\si{\planckbar} in \si{\electronvolt\second},
and $c$ in \si{\angstrom\per\second}.
Thus when written as $(c / ℏ) P = p E_P$,
where $p$ is a unitless function of the energy,
the important overall energy scale is
$E_P = at m_0 / (c ℏ) × 10^3 \si{\giga\electronvolt}$,
where the symbols in $E_p$ are the numerical magnitudes of the
quantities when expressed in the assumed units above.
In particular,
$E_P = at · \SI{0.259}{\giga\electronvolt}$,
and for $at = 3.2$, $E_p = \SI{0.83}{\giga\electronvolt}$.
Alternatively, one may write this in terms of fundamental constants:
using $at = 3.2$ again gives
$P = \left( 1.624 × 10^3 c ℏ m_0 \right) p$.
