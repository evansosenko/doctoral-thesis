\label{s:appendix:interaction}
The BCS theory of superconductivity is often introduced as
a consequence of the effective electron-electron interaction
arising from the underlying electron-phonon interaction.
In this appendix, we review general two-particle interactions,
and derive the form of the electron-electron interaction
when applied to the tight binding model.
These results are the foundation for deriving the
intrinsic TMD superconducting state.

\section{Electron-Electron Interaction}

Consider a general spin-independent interaction potential
$v \of{\vc{x}, \vc{x}'}$ which depends on the positions
$\vc{x}$ and $\vc{x}'$ of a pair of particles
\cite{ballentine1998quantum}.
In a multiparticle system of dimension $d$ with no self-interaction,
the interaction $V$ may be written as an additive pair operator
in a given bases of fermionic Fock operators,
$C_α^† \Ket{0} = \Ket{α} = \Ket{f_α} ⊗ \Ket{σ_α}$,
\begin{equation}
  \label{eq:interaction:general}
  \begin{aligned}
  V
  & = \frac{1}{2} ∑_{i ≠ j} v \of{\vc{Q}_i, \vc{Q}_j}, \\
  & = \frac{1}{4} \sumInt
    \Braket{α β | v | γ δ} C_α^† C_β^† C_δ C_γ, \\ % chktex 19
  & = \frac{1}{2} \sumInt
    v_{α β, γ δ} C_α^† C_β^† C_δ C_γ. % chktex 19
  \end{aligned}
\end{equation}
Here, $\Ket{α β} = \left( \Ket{α} \Ket{β}
- \Ket{β} \Ket{α} \right) / \sqrt{2}$,
and we introduce the symbol
\begin{align}
  v_{α β, γ δ} % chktex 19
  & = \left( \Bra{f_α} \Bra{f_β} \right) v
      \left( \Ket{f_γ} \Ket{f_δ} \right) % chktex 19
      \Braket{σ_α | σ_γ} \Braket{σ_β | σ_δ}, \\ % chktex 19
  & = δ_{σ_α σ_γ} δ_{σ_β σ_δ} % chktex 19
      ∫ \cc{f}_α \of{\vc{x}} \cc{f}_β \of{\vc{x}'} v % chktex 19
      \of{\vc{x}, \vc{x}'} f_γ \of{\vc{x}} f_δ \of{\vc{x}'} % chktex 19
      \dif^d \vc{x} \dif^d \vc{x}',
\end{align}
so that $\Braket{α β | v | γ δ} = v_{α β, γ δ} - v_{α β, δ γ}$. % chktex 19
For a given interaction, the explicit form of $V$
depends on the choice of basis states.
For illustration, we consider
the position and momentum space descriptions below.

If one chooses the continuous basis of position eigenstates,
$Ψ_σ^† \of{\vc{x}} \Ket{0} = \Ket{\vc{x}} ⊗ \Ket{σ}$,
then \cref{eq:interaction:general} takes the form of an integral over
density-density interactions,
\begin{equation}
  \label{eq:interaction:position-space}
  V
  = \frac{1}{2}
    ∫ v \of{\vc{x}, \vc{x}'}
    \normalorder{ρ \of{\vc{x}} ρ \of{\vc{x}'}}
    \dif^d \vc{x} \dif^d \vc{x}',
\end{equation}
where the colon denotes normal ordering and
\begin{equation}
  ρ \of{\vc{x}} = ∑_σ Ψ_σ^† \of{\vc{x}} Ψ_σ \of{\vc{x}}.
\end{equation}

For interactions which depend only on the relative separation,
$v \of{\vc{x}, \vc{x}'} = v \of{\vc{x}' - \vc{x}} = v \of{\vc{r}}$,
if one chooses the countable basis
of box normalized momentum eigenstates in a volume $Ω$
with periodic boundary conditions,
$c_{\vc{p} σ}^† \Ket{0} = \Ket{\vc{p}} ⊗ \Ket{σ}$,
with
$\Braket{\vc{x} | \vc{p}} = Ω^{-1 / 2} e^{i \vc{p} · \vc{x}}$
then
\begin{equation}
  \label{eq:interaction:momentum-space}
  V
  = \frac{1}{2}
    ∑_{\vc{p}, \vc{p'}, \vc{q}}
    ∑_{σ, σ'}
    \tilde{v}_{\vc{q}}
    c_{\vc{p} + \vc{q} σ}^† c_{\vc{p}' - \vc{q} σ'}^†
    c_{\vc{p}' σ'} c_{\vc{p} σ},
\end{equation}
where
\begin{equation}
  \label{eq:interaction:fourier:coefficients}
  \tilde{v}_{\vc{q}}
  = \frac{1}{Ω} ∫ v \of{\vc{r}} e^{-i \vc{q} · \vc{r}} \dif^d \vc{r}.
\end{equation}

\subsection{Interaction in the Tight Binding Model}

We now apply the above to the basis introduced in the tight binding model.
Introduce a Wannier representation,
\begin{equation}
  \Ket{ϕ_{ν \vK}}
  = \frac{1}{\sqrt{N}}
    ∑_{\vR} e^{i \vK ⋅ \vR}
    \Ket{ϕ_{ν \vR}},
\end{equation}
where for convenience we write $\vR$ for $\vRn{n}$.
We adopt the following section-specific notation for the Fock operators,
\begin{subequations}
  \begin{alignat}{2}
    a_{\vK ν σ}^† \Ket{0}
    &  = \Ket{ϕ_{ν \vK σ}}
    && = \Ket{ϕ_{ν \vK}} ⊗ \Ket{σ}, \\
    a_{\vR ν σ}^† \Ket{0}
    &  = \Ket{ϕ_{ν \vR σ}}
    && = \Ket{ϕ_{ν \vR}} ⊗ \Ket{σ}.
  \end{alignat}
\end{subequations}
which explicitly separates out the spin index.
Since
\begin{equation}
  \Braket{\vc{x} | ϕ_{ν \vR}}
  = \Braket{\vc{x} - \vR | φ_ν}
  = φ_ν \of{\vc{x} - \vR},
\end{equation}
by keeping only on-center, like-orbital terms,
the interaction may by simplified to the approximate form
\begin{equation}
  V
  ⋍ \frac{1}{2}
    ∑_{\vR, \vR'}
    ∑_{ν, ν'}
    v_{\vR \vR'}^{ν ν'}
    \normalorder{n_{\vR ν} n_{\vR' ν'}},
\end{equation}
with the number operators,
\begin{equation}
  n_{ν \vR}
  = ∑_σ a_{\vR ν σ}^† a_{\vR ν σ},
\end{equation}
and interaction integral,
\begin{equation}
  v_{\vR \vR'}^{ν ν'}
  = ∫ v \of{\vc{x}, \vc{x}'}
    \abs{φ_ν \of{\vc{x} - \vR}}^2
    \abs{φ_{ν'} \of{\vc{x}' - \vR'}}^2
    \dif^d \vc{x} \dif^d \vc{x}'.
\end{equation}
This sum over density-density interactions mimics
\cref{eq:interaction:position-space}.

Further, assuming that the interaction
depends only on the relative separation,
$v_{\vR \vR'}^{ν ν'} = v_{ν ν'} \of{\vR' - \vR}$,
we may introduce the Fourier expansion
\begin{equation}
  \tilde{v}^{ν ν'}_{\vc{q}}
  = ∑_{\vR} v_{ν ν'} \of{\vR} e^{i \vc{q} · \vR},
\end{equation}
and switch back to the Bloch representation,
\begin{multline}
  \label{eq:interaction:tight-binding:momentum}
  V
  = \frac{1}{2} {\left( \frac{Ω}{N} \right)}^2
    ∑_{ν, ν'}
    ∑_{σ, σ'}
    ∑_{\vK', \bar{\vK}}
    ∑_{\vK, \bar{\vK}'}
    ∑_{\vc{q}}
    \tilde{v}^{ν ν'}_{\vc{q}} \\
    δ \left[ \vc{q} - \left(\vK - \bar{\vK} \right) \right] % chktex 19
    δ \left[ \vc{q} - \left(\bar{\vK}' - \vK' \right) \right] % chktex 19
    a_{\bar{\vK} ν σ}^† a_{\bar{\vK}' ν' σ'}^†
    a_{\vK' ν' σ'} a_{\vK ν σ}.
\end{multline}
As a convenience, we have written this
as a sum over a countable set of momentum states,
however any sum over momentum may be converted to an integral
according to the substitution
$∑_{\vK} → \left( N / Ω \right) ∫ \dif^d \vK$.
Using this, we may integrate out the $δ$-functions % chktex 19
and obtain a form that mimics
\cref{eq:interaction:momentum-space},
\begin{equation}
  \label{eq:interaction:tight-binding:final}
  V
  = \frac{1}{2}
    ∑_{\vK, \vK', \vc{q}}
    ∑_{ν, ν'}
    ∑_{σ, σ'}
    \tilde{v}^{ν ν'}_{\vc{q}}
    a_{\vK + \vc{q} ν σ}^† a_{\vc{\vK' - \vc{q}} ν' σ'}^†
    a_{\vK' ν' σ'} a_{\vK ν σ}.
\end{equation}
