\label{s:appendix:tight-binding}
In this appendix,
we review the assumptions of the tight-binding model
as derived from Bloch's theorem.
We denote the position and momentum operators
respectively by $\vc{Q}$ and $\vc{P}$.

\section{Transition Operators}

We first define the position-space translation operator,
$T \of{\vc{r}} = e^{- i \vc{r} · \vc{P}}$,
and the momentum-space translation operator
$T_{\vK} = e^{- i \vc{k} · \vc{Q}}$.
We assume zero magnetic field.%
\footnote{%
  For nonzero field, much of the following can be recast
  in terms of the magnetic translation operators~\cite{PhysRev.133.A1038}.
}
Each has a simple inverse:
$T^{-1} \of{\vc{r}} = T \of{- \vc{r}}$
and
$T_{\vK}^{-1} = T_{- \vK}$.
We also note the derivatives:
$∇_{\vK} T_{\vK} = - i \vc{Q} T_{\vK}$
and
$∇_{\vK} T^{-1}_{\vK} = i \vc{Q} T^{-1}_{\vK}$.

We can compute a useful commutation relation of these translations operators.
Application of the Baker–Campbell–Hausdorff formula gives
\begin{subequations}
  \begin{align}
    T \of{\vc{r}} T_{\vK}
      & = e^{- i \left( \vc{r} · \vc{P} + \vc{k} · \vc{Q} \right)
          - \frac{1}{2} \left[ \vc{r} · \vc{P}, \vc{k} · \vc{Q} \right]}, \\
    %
    T_{\vK} T \of{\vc{r}}
      & = e^{- i \left( \vc{r} · \vc{P} + \vc{k} · \vc{Q} \right)
          + \frac{1}{2} \left[ \vc{r} · \vc{P}, \vc{k} · \vc{Q} \right]}.
  \end{align}
\end{subequations}
Only a single commutator appears above since
$\left[ \vc{r} · \vc{P}, \vc{k} · \vc{Q} \right] = - i \vc{r} · \vc{k}$.
Substituting this gives
\begin{equation}
  \label{eq:translations:commutation}
  T \of{\vc{r}} T_{\vK}
  = e^{-i \vc{r} · \vc{k}} T_{\vK} T \of{\vc{r}}.
\end{equation}

\section{Bloch Hamiltonian}

Given a Hamiltonian $H$ and a set of $N$ lattice vectors
$\left\{ \vRn{n} \right\}$ such that the Hamiltonian
commutes with each $T \of{\vRn{n}}$,
we may choose a set of common eigenvectors according to Bloch's theorem,
\begin{subequations}
  \begin{align}
    H \Ket{ψ_{n \vK}}
      & = E_{n \vK} \Ket{ψ_{n \vK}}, \\
    T \of{\vRn{m}} \Ket{ψ_{n \vK}}
      & = e^{-i \vK · \vRn{m}} \Ket{ψ_{n \vK}}.
  \end{align}
\end{subequations}
Given periodic boundary conditions,
the set of allowed $\vK$ becomes countable
and may be restricted to the first Billouin zone
due to the periodicity of the eigenvalues
with respect to translation by a reciprocal lattice vector $\vG$.
For each $\vK$, the transformation
$\Ket{u_{n \vK}} = T_{\vK} \Ket{ψ_{n \vK}}$
gives a set of states which are invariant under lattice translations, i.e.,
using \cref{eq:translations:commutation},
$T \of{\vRn{m}} \Ket{u_{n \vK}} = \Ket{u_{n \vK}}$.
A general operator then transforms according to
\begin{equation*}
  A_{\vK} = T_{\vK} A T_{\vK}^{-1}.
\end{equation*}
Note that the derivative of a transformed operator may be computed as
\begin{equation}
  ∇_{\vK} A_{\vK}
    = ∇_{\vK} \left( T_{\vK} A T_{\vK}^{-1} \right)
    = i \left[ A_{\vK}, \vc{Q} \right].
\end{equation}

\section{Tight Binding Approximation}

In the tight biding model, one assumes there exists a finite set
$\left\{ \Ket{φ_ν} \right\}$ of relevant atomic orbitals.
The corresponding Bloch orbital states,
\begin{equation}
  \Ket{ϕ_{ν \vK}}
  = \frac{1}{\sqrt{N}}
    ∑_{n = 1}^N e^{i \vK ⋅ \vRn{n}}
    T \of{\vRn{n}} \Ket{φ_ν},
\end{equation}
provide a complete bases for the space
of Bloch eigenstates $\left\{ \Ket{ψ_{n \vK}} \right\}$, i.e.,
\begin{subequations}
  \begin{align}
    \Ket{ψ_{n \vK}} & = ∑_ν M_{\vK}^{ν n} \Ket{ϕ_{ν \vK}}, \\
    \Ket{ϕ_{ν \vK}} & = ∑_n W_{\vK}^{n ν} \Ket{ψ_{n \vK}},
  \end{align}
\end{subequations}
with $M_{\vK}^{ν n} = \cc{W}_{\vK}^{ν n}$.
The second overlap term in
\begin{equation}
  \Braket{ϕ_{ν' \vK} | ϕ_{ν \vK}}
  = \Braket{φ_{ν'} | φ_ν}
    + ∑_{\vRn{n} ≠ 0}^N e^{i \vK ⋅ \vRn{n}}
    \Braket{φ_{ν'} | T \of{\vRn{n}} | φ_ν},
\end{equation}
is small, thus the states
$\left\{ \Ket{φ_ν} \right\}$ are assumed formally orthonormal.

Since each $\Ket{ϕ_{n k}}$ satisfies
\begin{equation}
  T \of{\vRn{m}} \Ket{ϕ_{ν \vK}}
  = e^{-i \vK · \vRn{m}} \Ket{ϕ_{ν \vK}},
\end{equation}
the transformed orbital states
$\Ket{v_{n \vK}} = T_{\vK} \Ket{ϕ_{n k}}$
are invariant under lattice translations.
One typically knows the matrix elements
of an effective Hamiltonian
in the bases of these periodic Bloch orbitals,
\begin{equation}
  H_{\vK}^{ν ν'}
  = \Braket{v_{ν \vK} | H_{\vK} | v_{ν' \vK}}
  = ∑_n \cc{W}_{\vK}^{n ν} E_{n \vK} W_{\vK}^{n ν'},
\end{equation}
which determines the coefficients $M_{\vK}^{ν n}$.
