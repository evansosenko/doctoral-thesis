\clearpage
\centering
\vspace*{-\toptafiddle}

\textsc{Abstract}

\thetitle{} \\
by \\
\theauthor{} \\

\thedegree, Graduate Program in \thefield{} \\
\theuniversity{} \\
\thechair{}, Chairperson \\
\thedate{}

\justify{}
Injection, transmission, and detection of spins in a conducting channel
are the basic ingredients of spintronic devices.
Long spin lifetimes during transit
are an important ingredient in realizing this technology.
An attractive platform for this purpose is graphene, which has high mobilities
and low spin-orbit coupling.
Unfortunately, measured spin lifetimes are orders of magnitude smaller
than theoretically expected.
A source of spin loss is the resistance mismatch between
the ferromagnetic electrodes and graphene.
While this has been studied numerically,
here we provide a closed form expression for Hanle spin precession
which is the standard method of measuring spin lifetimes.
This allows for a detailed characterization of the nonlocal spin valve device.

Two dimensional transition metal dichalcogenides entwine
interaction, spin-orbit coupling, and topology.
Hole-doped systems lack spin degeneracy:
states are indexed with spin and valley specificity.
This unique structure offers new possibilities
for correlated phases and phenomena.
We realize an unconventional superconducting pairing phase
which is an equal mixture of a spin singlet and the $m = 0$ spin triplet.
It is stable against large in-plane magnetic fields,
and its topology allows quasiparticle excitations
of net nonzero Berry curvature via pair-breaking circularly polarized light.

\enlargethispage{\bottafiddle}
\clearpage
