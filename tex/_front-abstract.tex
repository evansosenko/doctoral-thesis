\clearpage
\centering
\vspace*{-\toptafiddle}

\textsc{Abstract of the Dissertation} \\~\\

\SingleSpacing{}

\thetitle{} \\~\\
by \\~\\
\theauthor{} \\~\\

\thedegree, Graduate Program in \thefield{} \\
\theuniversity{}, \thedate{} \\
\thechair{} \\~\\

\justify{}
\DoubleSpacing{}
Recent focus on two dimensional materials and spin-coupled phenomena
holds future potential for fast, efficient, flexible, and transparent devices.

The fundamental operation of a spintronic device
depends on the injection, transmission, and detection
of spins in a conducting channel.
Long spin lifetimes during transit are critical for realizing this technology.
An attractive platform for this purpose is graphene, which has high mobilities
and low spin-orbit coupling.
Unfortunately, measured spin lifetimes are orders of magnitude smaller
than theoretically expected.
A source of spin loss is the resistance mismatch between
the ferromagnetic electrodes and graphene.
While this has been studied numerically,
here we provide a closed form expression for Hanle spin precession
which is the standard method of measuring spin lifetimes.
This allows for a detailed characterization of the nonlocal spin valve device.

Strong spin orbit interaction has the potential
to engender unconventional superconducting states.
A cousin to graphene, two dimensional transition metal dichalcogenides
entwine interaction, spin-orbit coupling, and topology.
The noninteracting electronic states have
multiple valleys in the energy dispersion
and are topologically nontrivial.
We report on the possible superconducting states
of hole-doped systems, and analyze to what extent the correlated phase
inherits the topological aspects of the parent crystal.
We find that local attractive interactions or proximal coupling to
$s$-wave superconductors lead to a pairing
which is an equal mixture of a spin singlet and the $m = 0$ spin triplet.
Its topology allows quasiparticle excitations
of net nonzero Berry curvature via pair-breaking circularly polarized light.
The valley contrasting optical response,
where oppositely circularly polarized light couples to different valleys,
is present even in the superconducting state but with smaller magnitude.

\enlargethispage{\bottafiddle}
\clearpage
