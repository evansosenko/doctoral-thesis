In this chapter we have analyzed the effect of contact resistance
on spin lifetimes determined via the Hanle spin prescession technique
in nonlocal spin valves.
The general expression for the precession curves given
in \cref{eq:f} is the main new result.
While aspects of the discussed phenomena have been addressed numerically before,
an analytic solution is obtained here
which allows for detailed characterization of the device.
In particular, general features of scaling
as well as various limits and regimes can be analyzed.
In addition, the solution allows for fitting data
using standard curve fitting algorithms.

In this letter, we report on the nature of the superconducting state
of hole-doped TMDs.
Remarkably, the correlated state inherits
the valley contrasting phenomena of the non-interacting state.
While the magnitude is smaller, pair-breaking produces quasiparticles
that have the same Berry curvature, and hence the same anomalous velocity.
Thus one predicts an anomalous Hall response unlike the valley Hall response
observed in \ce{MoSe2}.

While systematic synthesis and characterization of hole-doped systems
is still in its early stages, the fact that other two-dimensional compounds
and their bulk counterparts are known to be superconducting
\cite{%
  PhysRevB.88.054515%
}
provides impetus to explore this novel phenomena.
