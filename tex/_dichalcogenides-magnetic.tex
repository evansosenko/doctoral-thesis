\section{Effect of in-plane magnetic field}

We briefly discuss the pair-breaking effects
of an in-plane magnetic field and non-magnetic disorder.
(The details are presented elsewhere %
\footnote{%
  Junhua Zhang, Evan Sosenko, and Vivek Aji, in preparation.}.)
The superconducting state exhibits an anomalous magnetic response
due to the large spin-orbit interaction and spin splitting $2λ$.
Unlike conventional superconductors,
where a uniform magnetic field
leads to pair-breaking from spin paramagnetism,
here the Zeeman coupling from an in-plane magnetic field
in the clean system modifies the spin structure
of the Cooper pair and parametrically suppresses $T_c$.
Modification of the effective coupling, rather than pair-breaking,
causes this phenomena.
While, in compliance with Anderson's theorem,
the lack of pair-breaking for non-magnetic impurities
is recovered at zero field
(since time reversal symmetry is preserved),
the transition is indeed suppressed by a pair-breaking effect
from the combination of the magnetic field and scalar impurities.

The pair-breaking effect is characterized by the parameter
$δ_c
= τ_0^{-1} {\left( μ_B H_c^∥ / λ \right)}^2$,
where $μ_B$ is the Bohr magneton and $τ_0^{-1}$ is
the collision rate resulting from the scalar disorder potential.
Note that for a clean system, where $\tau_{0} \rightarrow \infty$,
we recover the result that there is no pair-breaking.
Assuming a continuous phase transition induced by the magnetic field,
the pair-breaking equation at temperature $T$ is
$\log{\of{T_c' / T}}
= ψ \of{1 / 2 + δ_c / {2 π k_B T}}
- ψ \of{1 / 2}$,
where $ψ \of{z}$ is the digamma function,
$T_c'$ is the transition temperature
in the clean system
(weakly suppressed by the in-plane field),
and $k_B$ is the Boltzmann constant.
This equation determines the upper critical field
$H_c^∥ \of{T}$ as a function of temperature.
Clearly, the upper critical field is greatly enhanced
by the large spin-orbit interaction.
At zero temperature, the upper
critical field is approximately
$μ_B H_c^∥ \of{0}
= λ \sqrt{2 π k_B T_c' τ_0 e^{ψ \of{1 / 2}}}$.
