\section{Introduction}

Spintronic devices rely on the ability to
inject, transport, manipulate, and detect spins
\cite{%
  Wolf16112001,%
  RevModPhys.76.323%
}.
The typical architecture involves ferromagnetic electrodes
deposited on a conducting medium
\cite{%
  1990ApPhL..56..665D,%
  Jedema2001%
}.
Driving a current across the junction of a magnetic element
and a nonmagnetic metal leads to spin injection (also called spin accumulation)
\cite{%
  PhysRevLett.55.1790,%
  Jedema2001,%
  Yang2008,%
  PhysRevLett.94.196601%
}.
The injected spins either diffuse in nonlocal spin valve geometry,
or are driven by applied fields across the conducting channel.
The former has the advantage that the observed spin signal
is not corrupted by accompanying charge current.
During this transit, scattering processes dephase the spins
and thus degrade the chemical potential imbalance
between spins of opposite orientation.
The residual difference is detected by a ferromagnetic electrode
whose magnetization can be flipped by applying external fields.

The performance of devices is determined by a number of parameters
associated with the basic processes described above.
The efficiency of spin injection, the diffusion length
(or equivalently the diffusion constant and spin relaxation time),
the distance between the injector and detector,
and resistivities of various components such as the electrodes,
the junction, and the conducting channel,
are some of the ingredients that contribute to the measured magnetoresistance.
As such, having good injection efficiency coupled with long spin lifetimes
is crucial for the viability of spintronic applications.
The discovery of graphene
\cite{Novoselov22102004}
has been of particular interest in this regard
because of its tunable conductivity, high mobility, and low spin-orbit coupling.
Moreover, the two dimensional nature allows
for efficient device design and spin manipulation.
Theoretical estimates for spin lifetimes of a few microseconds
\cite{%
  PhysRevB.74.155426,%
  Trauzettel2007%
}
are leading to a concerted effort in realizing
spin based transistors and spin valves
\cite{%
  Tombros2007,%
  JJAP.46.L605,%
  Cho2007,%
  PhysRevLett.101.046601,%
  1704408,%
  Han2012,%
  Han2012369,%
  PhysRevB.80.241403%
}.

Unfortunately, the best measured spin lifetimes
via the Hanle spin precession technique are in the
\SIrange{50}{200}{\pico \second} range
\cite{%
  PhysRevB.80.241403,%
  Tombros2007,%
  PhysRevB.80.214427,%
  PhysRevLett.104.187201%
}.
The large discrepancy is yet to be explained.
The linear scaling of spin and transport lifetimes
\cite{PhysRevB.80.241403}
suggested that the dominant scattering mechanism in the conducting channeling
is of the Elliot-Yafet
\cite{PhysRev.96.266}
type.
Surprisingly, in the regime of small spin lifetimes
($∼ \SI{100}{\pico \second}$),
Coulomb scattering was shown not to be the dominant mechanism
\cite{PhysRevLett.104.187201}.
The more important determining factor of the lifetime
was found to be the nature of the interface between
the magnetic electrode and the conducting channel.
Tunneling contacts suppress spin relaxation, and lifetimes of
\SI{771}{\pico \second}
were reported at room temperature, increasing to
\SI{1.2}{\nano \second} at \SI{4}{\kelvin}
\cite{PhysRevLett.107.047207}.
On the other hand, low resistance barriers lead to considerable
uncertainty in the determination of the lifetimes.

Over the last few years, characterizing the nature of the spin dynamics
at the interface has garnered much attention.
A key contribution in this effort is the generalization
of the standard theoretical approach
of calculating the nonlocal magnetoresistance
with and without the magnetic field.
Recent efforts study the effect of including the contact resistance
\cite{%
  PhysRevB.80.214427,%
  PhysRevB.67.052409%
},
and alternatively relaxing the normally infinite boundary conditions
in favor of a finite channel size
\cite{1404.6276v1}.
The approach relies on numerically solving the Bloch equation
to generate Hanle precession curves and then fitting observed data.

In this paper, we present the closed form expression
for the precession curves with finite contact resistance,
and analytically discuss the various parameters regimes
that show qualitatively different behaviors.
The fits to data reproduce the results in the literature
and provide a means to understand the effect of the contacts
which were previously obtained by numerical simulations.

The paper is organized as follows.
In \cref{s:model} we provide the basic model, define the relevant parameters,
and present an expression for the nonlocal resistance $\rNL$.
The primary result is given by \cref{eq:f}.
In \cref{s:fits} the solution for $\rNL$ is fitted to data.
In \cref{s:regimes} we analyze the various regimes which are determined by
the diffusion length, length of the device, and the contact resistance.
\Cref{s:summary} ends with a summary of the results and future directions.
