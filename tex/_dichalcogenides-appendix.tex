\section{Essential Physics}
\label{s:appendix}

In this appendix, we review the essential physical theory
underlying our model and computations.

\subsection{Optical Transitions}

Consider the spin-orbit Hamiltonian for a single non-interacting electron
which includes a position dependent potential $W \of{\vc{Q}}$
and an electromagnetic potential $\vc{A} \of{\vc{Q}}$,
\begin{equation}
  H = \frac{{\left( \vc{P} + e \vc{A} \right)}^2}{2 M}
      + λ \vc{L} · \vc{S} + W.
\end{equation}
Given $\vc{L} = \vc{Q} × \vc{P}$, one can show, either via
the commutation relations or formal differentiation,
that the velocity operator is
\begin{equation}
  \vc{V}
  = i [H, \vc{Q}] = \pderiv{H}{\vc{P}}
  = \frac{\vc{P} + e \vc{A}}{M} + λ \vc{Q} × \vc{S}.
\end{equation}

In a gauge with $∇ ·\vc{A} = 0$,
\begin{equation}
  H = \frac{\vc{P}^2}{2 M} + \frac{e}{M} \vc{A} · \vc{P} + e^2 \vc{A}^2.
\end{equation}

For monochromatic optical perturbations
with wave vector $\vc{q}$, frequency $ω$, and polarization vector $\vc{ϵ}$,
the electromagnetic potential is of the form
$\vc{A} = 2 \re{\vc{ϵ} A_0 e^{i \left( \vc{q} · \vc{Q} - ω t \right)}}$.
For linear perturbations, we neglect the term of order $\vc{A}^2$
to write the Hamiltonian in the form
$H = H_0 + H' e^{- i ω t} + {H'}^† e^{i ω t}$.
Here,
$H' = e A_0 e^{i \vc{q} · \vc{Q}} \left( \vc{ϵ} · \vc{V} \right)$
appears as a standard harmonic perturbation where
$\vc{V} = i [ H_0, \vc{Q} ]$
is the velocity operator for the unperturbed system.

Thus, according to Fermi's golden rule,
the optical transition probability per unit time is
\begin{equation}
  Γ_{f ← i}
  = 2 π e^2 A_0^2
    \abs{\Braket{f | e^{i \vc{q} · \vc{Q}} \vc{ϵ} · \vc{V} | i}}^2
    δ \of{E_f + E_i - ω}. % chktex 19
\end{equation}
Typically, the dipole approximation is used in which
$e^{i \vc{q} · \vc{Q}} → 1$.

\subsection{Transition Operators}

We first define the position-space translation operator,
$T \of{\vc{r}} = e^{- i \vc{r} · \vc{P}}$,
and the momentum-space translation operator
$T_{\vK} = e^{- i \vc{k} · \vc{Q}}$.
We assume zero magnetic field.%
\footnote{%
  For nonzero field, however, much of the following can be recast
  in terms of the magnetic translation operators.
  % TODO Add references.
}
Each has a simple inverse:
$T^{-1} \of{\vc{r}} = T \of{- \vc{r}}$
and
$T_{\vK}^{-1} = T_{- \vK}$.
We also note the derivatives:
$∇_{\vK} T_{\vK} = - i \vc{Q} T_{\vK}$
and
$∇_{\vK} T^{-1}_{\vK} = i \vc{Q} T^{-1}_{\vK}$.

We can compute a useful commutation relation of these translations operators.
Application of the Baker–Campbell–Hausdorff formula gives
\begin{subequations}
  \begin{align}
    T \of{\vc{r}} T_{\vK}
      & = e^{- i \left( \vc{r} · \vc{P} + \vc{k} · \vc{Q} \right)
          - \frac{1}{2} \left[ \vc{r} · \vc{P}, \vc{k} · \vc{Q} \right]}, \\
    %
    T_{\vK} T \of{\vc{r}}
      & = e^{- i \left( \vc{r} · \vc{P} + \vc{k} · \vc{Q} \right)
          + \frac{1}{2} \left[ \vc{r} · \vc{P}, \vc{k} · \vc{Q} \right]}.
  \end{align}
\end{subequations}
Only a single commutator appears above since
$\left[ \vc{r} · \vc{P}, \vc{k} · \vc{Q} \right] = - i \vc{r} · \vc{k}$,
and substituting this in above we see
\begin{equation}
  \label{eq:translations:commutation}
  T \of{\vc{r}} T_{\vK}
  = e^{-i \vc{r} · \vc{k}} T_{\vK} T \of{\vc{r}}.
\end{equation}

\subsection{Bloch Hamiltonian}

Given a Hamiltonian $H$ and a set of $N$ lattice vectors
$\left\{ \vRn{n} \right\}$ such that the Hamiltonian
commutes with each $T \of{\vRn{n}}$,
we may choose a set of common eigenvectors according to Bloch's theorem,
\begin{subequations}
  \begin{align}
    H \Ket{ψ_{n \vK}}
      & = E_{n \vK} \Ket{ψ_{n \vK}}, \\
    T \of{\vRn{m}} \Ket{ψ_{n \vK}}
      & = e^{-i \vK · \vRn{m}} \Ket{ψ_{n \vK}}.
  \end{align}
\end{subequations}
Given periodic boundary conditions,
the set of allowed $\vK$ becomes countable
and may be restricted to the first Billouin zone
due to the periodicity of the eigenvalues
with respect to translation by a reciprocal lattice vector $\vG$.
For each $\vK$, the transformation
$\Ket{u_{n \vK}} = T_{\vK} \Ket{ψ_{n \vK}}$
gives a set of states which are invariant under lattice translations, i.e.,
using \cref{eq:translations:commutation},
$T \of{\vRn{m}} \Ket{u_{n \vK}} = \Ket{u_{n \vK}}$.
A general operator then transforms according to
$A_{\vK}~=~T_{\vK} A T_{\vK}^{-1}$.
In particular,
\begin{align}
  H_{\vK}
    & = ∑_{n \vK'} E_{n \vK'} T_{\vK} \Ket{ψ_{n \vK'}}
        \Bra{ψ_{n \vK'}} T_{\vK}^{-1}, \\
    & = H_{\vK}^0 + H_{\vK}^1,
\end{align}
where
\begin{subequations}
  \begin{align}
    \label{eq:a:h.bloch.terms}
    H_{\vK}^0
      & = ∑_n E_{n \vK} \Ket{u_{n \vK}} \Bra{u_{n \vK}}, \\
    H_{\vK}^1
      & = ∑_n ∑_{\vK' ≠ \vK} E_{n \vK'} T_{\vK - \vK'}
          \Ket{u_{n \vK'}} \Bra{u_{n \vK'}} T_{\vK - \vK'}^{-1}.
  \end{align}
\end{subequations}

Note that the derivative of a transformed operator may be computed as
\begin{equation}
  ∇_{\vK} A_{\vK}
    = ∇_{\vK} \left( T_{\vK} A T_{\vK}^{-1} \right)
    = i \left[ A_{\vK}, \vc{Q} \right].
\end{equation}
In particular,
\begin{equation}
  ∇_{\vK} H_{\vK} = \vc{V}_{\vK}.
\end{equation}
Thus, for each $\vK$,
the matrix element appearing in the optical transition rate
(in the dipole approximation) may be computed as
\begin{multline}
  \Braket{ψ_{n' \vK} | \vc{V} | ψ_{n \vK}}
  = \Braket{u_{n' \vK} | \vc{V}_{\vK} | u_{n \vK}} \\
  = \Braket{u_{n' \vK} | ∇_{\vK} H_{\vK} | u_{n \vK}}.
\end{multline}

\subsection{Tight Binding Model}

In the tight biding model, one assumes there exists a finite set
$\left\{ \Ket{φ_ν} \right\}$ of relevant atomic orbitals.
The corresponding Bloch orbital states,
\begin{equation}
  \Ket{ϕ_{ν \vK}}
  = \frac{1}{\sqrt{N}}
    ∑_{n = 1}^N e^{i \vK ⋅ \vRn{n}}
    T \of{\vRn{n}} \Ket{φ_ν},
\end{equation}
provide a complete bases for the space
of Bloch eigenstates $\left\{ \Ket{ψ_{n \vK}} \right\}$, i.e.,
\begin{subequations}
  \begin{align}
    \Ket{ψ_{n \vK}} & = ∑_ν M_{\vK}^{ν n} \Ket{ϕ_{ν \vK}}, \\
    \Ket{ϕ_{ν \vK}} & = ∑_n W_{\vK}^{n ν} \Ket{ψ_{n \vK}},
  \end{align}
\end{subequations}
with $M_{\vK}^{ν n} = \cc{W}_{\vK}^{ν n}$.
The second overlap term in
\begin{multline}
  \Braket{ϕ_{ν' \vK} | ϕ_{ν \vK}}
  = \Braket{φ_{ν'} | φ_ν} \\
    + ∑_{\vRn{n} ≠ 0}^N e^{i \vK ⋅ \vRn{n}}
    \Braket{φ_{ν'} | T \of{\vRn{n}} | φ_ν},
\end{multline}
is small, thus the states
$\left\{ \Ket{φ_ν} \right\}$ are assumed formally orthonormal.

Since each $\Ket{ϕ_{n k}}$ satisfies
\begin{equation}
  T \of{\vRn{m}} \Ket{ϕ_{ν \vK}}
  = e^{-i \vK · \vRn{m}} \Ket{ϕ_{ν \vK}},
\end{equation}
the transformed orbital states
$\Ket{v_{n \vK}} = T_{\vK} \Ket{ϕ_{n k}}$
are invariant under lattice translations.
One typically knows the matrix elements
of an effective Hamiltonian
in the bases of these periodic Bloch orbitals,
\begin{equation}
  H_{\vK}^{ν ν'}
  = \Braket{v_{ν \vK} | H_{\vK} | v_{ν' \vK}}
  = ∑_n \cc{W}_{\vK}^{n ν} E_{n \vK} W_{\vK}^{n ν'},
\end{equation}
which determines the coefficients $M_{\vK}^{ν n}$.

\subsubsection{Optical Matrix Elements}

The matrix elements of the derivative,
$\Braket{u_{n \vK} | ∇_{\vK} H_{\vK} | u_{n' \vK}}$,
appearing in the optical translation rate may be cumbersome to compute.
We show that in the tight binding approximation with only $d$-type orbitals,
it is sufficient to compute the derivative of the orbital matrix elements,
$∇_{\vK} H_{\vK}^{ν ν'}
= ∇_{\vK} \Braket{v_{ν \vK} | H_{\vK} | v_{ν' \vK}}$.

We may write \cref{eq:a:h.bloch.terms} as
\begin{align}
  \label{appendix:optical:decomposition}
  H_{\vK}^0
  & = ∑_{ν, ν'}
      H_{\vK}^{ν ν'}
      \Ket{v_{ν \vK}} \Bra{v_{ν' \vK}}, \\
  H_{\vK}^1
  & = ∑_{ν, ν'} ∑_{\vK' ≠ \vK}
      H_{\vK'}^{ν ν'}
      T_{\vK - \vK'}
      \Ket{v_{ν \vK'}} \Bra{v_{ν' \vK'}}
      T_{\vK - \vK'}^{-1}.
\end{align}
First, using
$∇_{\vK} H_{\vK}^1 = i \left[ H_{\vK}^1, \vc{Q} \right]$,
we obtain a sum over $\vK' ≠ \vK$ of terms proportional
to $δ_{\vK, \vK'}$, % chktex 19
thus all matrix elements for $∇_{\vK} H_{\vK}^1$ vanish.
Next, we have
\begin{multline}
  \label{appendix:optical:sum}
  ∇_{\vK} H_{\vK}^0
  = ∑_{ν, ν'}
    ∇_{\vK} H_{\vK}^{ν ν'}
    \Ket{v_{ν \vK}} \Bra{v_{ν' \vK}} \\
  + ∑_{ν, ν'}
    H_{\vK}^{ν ν'}
    \left( ∇_{\vK} \Ket{v_{ν \vK}} \right) \Bra{v_{ν' \vK}} \\
  + ∑_{ν, ν'}
    H_{\vK}^{ν ν'}
    \Ket{v_{ν \vK}} \left( ∇_{\vK}  \Bra{v_{ν' \vK}} \right).
\end{multline}
Using
\begin{equation}
  \label{appendix:optical:ket:del}
  ∇_{\vK} \Ket{v_{ν \vK}}
  = \frac{T_{\vK}}{i \sqrt{N}}
    ∑_{n = 1}^N e^{i \vK ⋅ \vRn{n}}
    T \of{\vRn{n}} \vc{Q} \Ket{φ_ν},
\end{equation}
we find the second sum in
\cref{appendix:optical:sum}
contains terms proportional to the local optical matrix elements,
$\Braket{φ_ν' | T \of{\vRn{n}} \vc{Q} | φ_ν}$.
For finite $\vRn{n}$, these off-center integrals are small,
and for $\vRn{n} = 0$, optical transitions between $d$-orbitals
are forbidden by symmetry.
Similar logic applies to the third sum in
\cref{appendix:optical:sum},
thus
\begin{equation}
  \label{appendix:optical:approx}
  \Braket{u_{n \vK} | ∇_{\vK} H_{\vK} | u_{n' \vK}}
  ⋍ ∑_{ν, ν'}
    \cc{M}_{τ σ}^{ν n} \ofK
    \left( ∇_{\vK} H_{\vK}^{ν ν'} \right)
    M_{τ σ}^{ν' n'} \ofK
\end{equation}
