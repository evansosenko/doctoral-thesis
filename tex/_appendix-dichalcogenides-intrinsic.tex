\section{Interaction in the Effective Model}

Note that for the effective low-energy, dual-valley model,
\cref{eq:interaction:tight-binding:final}
is oversimplified
We must first split up each integral into
a region about each valley center.
The allowed transitions are still constrained
by global conservation of momentum.

Starting with
\cref{eq:interaction:tight-binding:momentum},
the integral over $\vc{q}$ is unchanged,
\begin{multline}
  V
  = \frac{1}{2} \frac{Ω}{N}
    ∑_{ν, ν'}
    ∑_{σ, σ'}
    ∑_{\vK', \bar{\vK}}
    ∑_{\vK, \bar{\vK}'}
    \tilde{v}^{ν ν'}_{\vK - \bar{\vK}}
    δ \left[ % chktex 19
      \left(\bar{\vK}' - \vK' \right) - \left(\vK - \bar{\vK} \right)
    \right]
    a_{\bar{\vK} ν σ}^† a_{\bar{\vK}' ν' σ'}^†
    a_{\vK' ν' σ'} a_{\vK ν σ}.
\end{multline}
We now split up the integral over the global momentum coordinates
into integrals over relative coordinates centered about each valley;
this introduces an additional overall factor of $2^{-4}$
(two valley centers per momentum integral with
four independent momentum-space coordinates).
We then restrict each integral to a suitable region about each valley
(indicated by a prime on the summation).
The relative coordinates are thus introduced by the substitution
$\vK → \vK + τ \vc{K}$, etc.

Global conservation of momentum, represented by the $δ$-function, % chktex 19
now requires
\begin{equation}
  \vK - \bar{\vK} + \left( τ - \bar{τ} \right) \vc{K}
  = \bar{\vK}' - \vK' + \left( \bar{τ}' - τ' \right) \vc{K}.
\end{equation}
Since $\abs{\vK} ≪ \abs{\vc{K}}$,
the above is actually the two independent conditions,
\begin{subequations}
  \begin{align}
    \vK - \bar{\vK}
    & = \bar{\vK}' - \vK', \\
    τ - \bar{τ}
    & = \bar{τ}' - τ'.
  \end{align}
\end{subequations}
There are three allowed cases for the sum over the valley indexes:
intravalley scattering with $τ = \bar{τ}$ and $τ' = \bar{τ}'$;
intervalley scattering with $τ = - τ'$ and $\bar{τ} = - \bar{τ}'$;
and exchange with $τ = \bar{τ}'$ and $\bar{τ} = τ'$.
We indicate summation over the allowed cases
by adding a prime to the sum.
Thus, we obtain
\begin{multline}
  V
  = \frac{1}{2^5}
    ∑_{ν, ν'}
    ∑_{σ, σ'}
    \sideset{}{'} ∑_{\vK, \vK', \vc{q}}
    \sideset{}{'} ∑_{\substack{τ, \bar{τ}, \\ τ', \bar{τ}'}}
    \tilde{v}^{ν' ν}_{\vc{q} + \left( \bar{τ} - τ \right) \vc{K}} \\
    {a_{\bar{τ} σ}^ν} ^† \of{\vK + \vc{q}}
    {a_{\bar{τ}' σ'}^{ν'}}^† \of{\vK' - \vc{q}}
    a_{τ' σ'}^{ν'} \ofKP
    a_{τ σ}^ν \ofK.
\end{multline}

The expected BCS instability is strongest for scattering
with $\vK = - \vK'$.
Restricting the sum over $\vK'$ to this condition,
relabeling the momentum indexes,
and defining $v_{\vc{q}}^{ν ν'} = 2^{-4} \tilde{v}_{\vc{q}}^{ν' ν}$ gives
\begin{multline}
  V
  = \frac{1}{2}
    ∑_{ν, ν'}
    ∑_{σ, σ'}
    \sideset{}{'} ∑_{\vK, \vK'}
    \sideset{}{'} ∑_{\substack{τ, \bar{τ}, \\ τ', \bar{τ}'}}
    v^{ν ν'}_{\vK' - \vK + \left( \bar{τ} - τ \right) \vc{K}} \\
    {a_{\bar{τ} σ}^ν}^† \ofKP
    {a_{\bar{τ}' σ'}^{ν'}}^† \ofMKP
    a_{τ' σ'}^{ν'} \ofMK
    a_{τ σ}^ν \ofK.
\end{multline}
Expanding the sum over the valley indexes gives
\begin{multline}
  V
  = \frac{1}{2}
    \sideset{}{'} ∑_{\vK, \vK'}
    ∑_{τ, τ',}
    ∑_{σ, σ'}
    ∑_{ν, ν'}
    [
    v^{ν ν'}_{\vK' - \vK}
    {a_{τ σ}^ν}^† \ofKP
    {a_{τ' σ'}^{ν'}}^† \ofMKP
    a_{τ' σ'}^{ν'} \ofMK
    a_{τ σ}^ν \ofK
    \\ +
    v^{ν ν'}_{\vK' - \vK + \left( τ' - τ \right) \vc{K}}
    {a_{τ' σ}^ν}^† \ofKP
    {a_{-τ' σ'}^{ν'}}^† \ofMKP
    a_{-τ σ'}^{ν'} \ofMK
    a_{τ σ}^ν \ofK
    \\ +
    v^{ν ν'}_{\vK' - \vK + \left( τ' - τ \right) \vc{K}}
    {a_{τ' σ}^ν}^† \ofKP
    {a_{τ σ'}^{ν'}}^† \ofMKP
    a_{τ' σ'}^{ν'} \ofMK
    a_{τ σ}^ν \ofK
    ].
\end{multline}

Projecting to bands with $α = τ = σ$,
\begin{multline}
  V
  = \frac{1}{2}
    \sideset{}{'} ∑_{\vK, \vK'}
    ∑_{ν, ν'}
    ∑_α
    v^{ν ν'}_{\vK' - \vK}
    [
    {a^ν_α}^† \ofKP
    {a^{ν'}_α}^† \ofMKP
    a^{ν'}_α \ofMK
    a^ν_α \ofK
    \\ +
    {a^ν_α}^† \ofKP
    {a^{ν'}_{-α}}^† \ofMKP
    a^{ν'}_{-α} \ofMK
    a^ν_α \ofK
    \\ +
    ∑_{α'}
    {a^ν_{α}}^† \ofKP
    {a^{ν'}_{α'}}^† \ofMKP
    a^{ν'}_{α'} \ofMK
    a^ν_α \ofK
    ].
\end{multline}
This simplifies into explicit intervalley and intravalley terms,
\begin{multline}
  \label{eq:interaction:tight-binding:superconducting}
  V
  = \sideset{}{'} ∑_{\vK, \vK'}
    ∑_{ν, ν'}
    ∑_α
    v^{ν ν'}_{\vK' - \vK}
    [
    {a^ν_α}^† \ofKP
    {a^{ν'}_α}^† \ofMKP
    a^{ν'}_α \ofMK
    a^ν_α \ofK
    \\ +
    {a^ν_α}^† \ofKP
    {a^{ν'}_{-α}}^† \ofMKP
    a^{ν'}_{-α} \ofMK
    a^ν_α \ofK
    ].
\end{multline}

\section{Superconducting Channels}

Assuming the interaction is real-valued and orbital-independent,%
\footnote{%
  As noted in \cref{s:dichalcogenides:intrinsic},
  this choice forbids the intravalley pairing.
  However, for the complex-valued case, only the intravalley pairing
  term survives.
  For completeness, we still consider the intravalley terms in this appendix.
}
$v^{ν ν'}_{\vK' - \vK} = v \of{\vK' - \vK} = v \of{\vK - \vK'}$,
\cref{eq:interaction:tight-binding:superconducting}
projected to the upper $n = -1$ bands with $τ = σ$ is
\begin{multline}
  \label{eq:intrinsic:projected}
  P_{τ = σ}^{-} \left( H^V \right) =
    \sumKK v \of{\vK' - \vK}
    \\ × \left(
      A_{\vK \vK'}^2 c_{\vK' ↑}^† c_{-\vK' ↑}^† c_{-\vK ↑} c_{\vK ↑}
      + \cc{A}_{\vK \vK'}^2 c_{\vK' ↓}^† c_{-\vK' ↓}^† c_{-\vK ↓} c_{\vK ↓}
    \right. \\ + \left.
      2 \abs{A_{\vK \vK'}}^2
      c_{\vK' ↑}^† c_{-\vK' ↓}^† c_{-\vK ↓} c_{\vK ↑}
    \vphantom{2 \abs{A_{\vK \vK'}}^2} \right),
\end{multline}
where
\begin{equation}
  A_{\vK \vK'} =
    e^{i \left( ϕ_{\vK'} - ϕ_{\vK} \right)}
    \sin{\frac{θ_{\vK'}}{2}} \sin{\frac{θ_{\vK}}{2}}
    + \cos{\frac{θ_{\vK'}}{2}} \cos{\frac{θ_{\vK}}{2}}.
\end{equation}

For the intravalley channels, the coefficient can be expanded as
\begin{equation}
  A_{\vK \vK'}^2
  = \sum_{m = 0}^2 \cc{f}_{\vK}^m · g_{\vK'}^m,
\end{equation}
with $f_{\vK}^m = g_{\vK'}^m$ and
\begin{subequations}
  \begin{align}
    f_{\vK}^0
    & = \cos^2{\frac{θ_{\vK}}{2}}
      = \frac{1}{2} P_0 \of{\cos{θ_{\vK}}}
      + \frac{1}{2} P_1 \of{\cos{θ_{\vK}}}, \\
    e^{-i ϕ_{\vK}} f_{\vK}^1
    & = \sqrt{2} \sin{\frac{θ_{\vK}}{2}} \cos{\frac{θ_{\vK}}{2}}
      = \frac{1}{\sqrt{2}} P_1 \of{\sin{θ_{\vK}}}, \\
    e^{-2 i ϕ_{\vK}} f_{\vK}^2
    & = \sin^2{\frac{θ_{\vK}}{2}}
      = \frac{1}{2} P_0 \of{\cos{θ_{\vK}}}
      - \frac{1}{2} P_1 \of{\cos{θ_{\vK}}}.
  \end{align}
\end{subequations}
Here, $P_l$ are the Legendre polynomials:
$P_0 \of{x} = 1$ and $P_1 \of{x} = x$.

For the intervalley channels, the coefficient can be expanded as
\begin{equation}
  2 \abs{A_{\vK \vK'}}^2
  = \sum_{l = 0}^1 \cc{f}_{\vK}^l · g_{\vK'}^l
  + \cc{f}_{\vK} · g_{\vK'},
\end{equation}
with $f_{\vK}^l = g_{\vK'}^l$,
$f_{\vK} = g_{\vK'}$, and
\begin{subequations}
  \begin{align}
    f_{\vK}^0
    & = \sqrt{2} P_0 \of{\cos{θ_{\vK}}}, \\
    f_{\vK}^1
    & = \sqrt{2} P_1 \of{\cos{θ_{\vK}}}, \\
    f_{\vK}
    & = \sqrt{2} P_1 \of{\sin{θ_{\vK}}} \vc{\hat{k}}.
  \end{align}
\end{subequations}

\section{Mean Field Approximation}

Using the mean field approximation, we make replacements of the form
$A B = A \ev{B} + \ev{A} B - \ev{A} \ev{B}$,
where $A$ ($B$) is the product of two creation (annihilation) operators.
The expectation value is taken in the superconducting ground state.
We assume $v \of{ \vK - \vK'} = - v_0$ is a constant attractive interaction,
possibly with some effective interaction range
which further restricts the summation.
\Cref{eq:intrinsic:projected} thus reduces to a sum of terms of the form
\begin{equation}
  - \sumK \left(
    \cc{Δ}_{\vK}^{γ γ'} c_{-\vK γ'} c_{\vK γ}
    + \frac{ε_{γ γ'}}{2}
  \right) + \hc,
\end{equation}
where $γ = γ' = ±1$ ($γ = - γ' = 1$) corresponds to the
intravalley (intervalley) scattering channels,
\begin{equation}
  Δ_{\vK}^{γ γ'} =
    - \sumKP \cc{v}_{\vK \vK'}^{γ γ'} \ev{c_{-\vK' γ'} c_{\vK' γ}},
\end{equation}
and
\begin{align}
  ε_{γ γ'}
    & = - \sumKK v_{\vK \vK'}^{γ γ'}
        \ev{c_{\vK' γ}^† c_{-\vK' γ'}^†} \ev{c_{-\vK γ'} c_{\vK γ}}, \\
    & = \sumK Δ_{\vK}^{γ γ'} \ev{c_{\vK γ}^† c_{-\vK γ'}^†}.
\end{align}

Projected to a single superconducting channel, the Hamiltonian is
\begin{multline}
  \label{eq:intrinsic:projected:channel}
  P_{γ γ'}^{-} \left( H^0 + H^V - μ N \right) = \\
    ε_{γ γ'}
    + \sumK \left(
      ξ_{\vK} c_{\vK γ}^† c_{\vK γ}
      + δ_{γ, -γ'} ξ_{\vK} c_{\vK γ'}^† c_{\vK γ'} % chktex 19
    \right) \\
    - \sumK \left(
    \cc{Δ}_{\vK}^{γ γ'} c_{-\vK γ'} c_{\vK γ}
    + Δ_{\vK}^{γ γ'} c_{\vK γ}^† c_{-\vK γ'}^†
    \right).
\end{multline}

\section{Channel Solutions}

An interaction for a given channel may be written as
\begin{equation}
  v_{\vK \vK'} = \cc{v}_{\vK' \vK} = - v_0 \cc{f}_{\vK} · g_{\vK'},
\end{equation}
where we suppress the channel and band indexes here and in the following
when there is no ambiguity.
The channels are further split according to angular momentum,
and the individual channels and their weights are summarized below.

The order parameter is
\begin{equation}
  \label{eq:intrinsic:gap}
  χ_0 = v_0 \sumK \cc{g}_{\vK} \ev{c_{-\vK γ'} c_{\vK γ}},
\end{equation}
thus $Δ_{\vK} = f_{\vK} · χ_0$.
Note that we allow $f_{\vK}$, $g_{\vK}$, and $χ_0$
to be either scalar or vector quantities.

The Hamiltonian in \cref{eq:intrinsic:projected:channel}
is again identical in structure to the BCS Hamiltonian,
however the parameter $Δ_{\vK}$ must be allowed complex
as multiple channels may differ by relative phases
which cannot be removed by a global unitary transformation.
The solutions for each channel all share an identical form;
however, for the intravalley channels,
the expression for the eigenvalues $λ_{\vK}$
and other associated expressions
is modified according to $ξ → ξ / 2$,
since the kinetic energy is split between each valley.
To keep track of the two cases, we will write $ξ'$,
where $ξ' = ξ$ for intervalley channels and
$ξ' = ξ / 2$ for intravalley channels.

The diagonalized form is
\begin{equation}
  P \left( H^0 + H^V - μ N \right)
  = \sumK ∑_{α = γ, γ'} λ_{\vK} b_{\vK α}^† b_{\vK α}
  + \sumK \left( ξ_{\vK}' - λ_{\vK} \right) + ε,
\end{equation}
with eigenvalues
\begin{equation}
  λ_{\vK} = \sqrt{ξ_{\vK}'^2 + \abs{Δ_{\vK}}^2}.
\end{equation}
The Bogoliubov transformation for complex $Δ_{\vK}$ is
\begin{subequations}
  \begin{align}
    c_{\vK γ}
    & = e^{-i δ_{\vK}} \cos{β_{\vK}} b_{\vK γ}       % chktex 19
      + e^{i δ_{\vK}'} \sin{β_{\vK}} b_{-\vK γ'}^†, \\ % chktex 19
    c_{-\vK γ'}
    & = e^{i δ_{\vK}'} \sin{β_{\vK}} b_{\vK γ}^† % chktex 19
      - e^{-i δ_{\vK}} \cos{β_{\vK}} b_{-\vK γ'},  % chktex 19
  \end{align}
\end{subequations}
where $δ_{\vK}' - δ_{\vK} = \arg{Δ_{\vK}}$ and % chktex 19
\begin{subequations}
  \begin{alignat}{3}
    \sin{2 β_{\vK}} & = && {}-{} && \abs{Δ_{\vK}} / λ_{\vK}, \\
    \cos{2 β_{\vK}} & = && {} {} && ξ_{\vK}' / λ_{\vK}.
  \end{alignat}
\end{subequations}
Note also the inverse,
\begin{subequations}
  \begin{align}
    b_{\vK γ}
    & = e^{i δ_{\vK}} \cos{β_{\vK}} c_{\vK γ}       % chktex 19
      + e^{i δ_{\vK}'} \sin{β_{\vK}} c_{-\vK γ'}^†, \\ % chktex 19
    b_{-\vK γ'}
    & = e^{i δ_{\vK}'} \sin{β_{\vK}} c_{\vK γ}^† % chktex 19
      - e^{i δ_{\vK}} \cos{β_{\vK}} c_{-\vK γ'}.  % chktex 19
  \end{align}
\end{subequations}

With this, \Cref{eq:intrinsic:gap} becomes
\begin{equation}
  \label{eq:intrinsic:gap:general}
  \begin{aligned}
    χ_0
    & = v_0 \sumK \cc{g}_{\vK}
        \left( - \frac{1}{2} e^{i \arg{Δ_{\vK}}} \sin{2 β_{\vK}} \right), \\
    & = \frac{v_0}{2} \sumK \cc{g}_{\vK}
        \frac{\abs{Δ_{\vK}} e^{i \arg{Δ_{\vK}}}}{λ_{\vK}}, \\
    & = \frac{v_0}{2} \sumK \cc{g}_{\vK}
        \frac{Δ_{\vK}}{λ_{\vK}}, \\
    & = \frac{v_0}{2} \sumK \cc{g}_{\vK}
        \frac{f_{\vK} · χ_0}{λ_{\vK}}.
  \end{aligned}
\end{equation}

\section{Gap equation}

We now derive an gap equation for each symmetry channel.
These are reduced to an integral which may be solved numerically.

\subsection{Scalar Channels}

For scalar channels, $f_{\vK} = g_{\vK}$.
We replace the sum by an integral,
and since $\abs{f_{\vK}}^2$ and $λ_{\vK}$ depend only on $\abs{\vK}$,
the integral over $ϕ$ is trival and yields a factor of $2 π$.
\Cref{eq:intrinsic:gap:general} becomes
\begin{equation}
  1
  = π v_0 ∫_{-ω}^{ω}
  \frac{\abs{f \of{ξ}}^2 \abs{ρ \of{ξ}} \dif ξ}
  {\sqrt{ξ'^2 + \abs{f \of{ξ}}^2 \abs{χ_0}^2}},
\end{equation}
where $ω < λ$ is the energy cutoff around the chemical potential
and the density of states is
\begin{equation}
  ρ \of{ξ}
  = k \pderiv{k}{ξ}
  = \frac{2 \left( ξ + μ \right) - λ}{{\left( a t \right)}^2}.
\end{equation}

\section{Vector Channels}

For the intervalley vector channels,
$f_{\vK} = g_{\vK} = \sqrt{2} \sin{θ_{\vK} \vc{\hat{k}}}$
and $χ_0 = \left( \abs{χ_0} / \sqrt{2} \right)
\left( \hat{e}_1 + i \hat{e}_2 \right)$
for some fixed unit vectors $\hat{e}_1$ and $\hat{e}_2$.
We consider two cases:
$\hat{e}_1 ∥ \hat{e}_2$ or $\hat{e}_1 ⊥ \hat{e}_2$.

When $\hat{e}_1 ∥ \hat{e}_2$,
write $χ_0 = \abs{χ_0} \hat{e} e^{i ϕ_0}$
and $\vc{\hat{k}} · \hat{e} = \cos{\left( ϕ_{\vK} - ϕ_e \right)}$
so \cref{eq:intrinsic:gap:general} reads
\begin{equation}
  \begin{aligned}
    \hat{e}
    & = \frac{v_0}{2} \sumK \frac{\abs{f_{\vK}}^2}{λ_{\vK}}
        \left( \hat{e} · \vc{\hat{k}} \right) \vc{\hat{k}}.
  \end{aligned}
\end{equation}
Dotting both sides with $\hat{e}$ and converting to integral form,
\begin{equation}
  1
  = \frac{v_0}{2} ∫_{-ω}^{ω} ∫_{0}^{2 π}
  \frac{\abs{f \of{ξ}}^2 \cos^2{ϕ} \abs{ρ \of{ξ}} \dif ϕ \dif ξ}
  {\sqrt{ξ^2 + \abs{f \of{ξ}}^2 \abs{χ_0}^2 \cos^2{ϕ}}},
\end{equation}
where as expected by symmetry,
the integral has been made independent of the direction of $\hat{e}$
through the substitution $ϕ → ϕ + ϕ_e$.
The integral over $ϕ$ can be
written in terms of elliptical functions using the identity
\begin{multline}
  \frac{a^2}{2} ∫_{0}^{2 π}
  \frac{\cos^2{ϕ} \dif ϕ}{\sqrt{1 + a^2 \cos^2{ϕ}}}
  = E \of{-a^2} - K \of{-a^2} \\
  + \sqrt{1 + a^2} E \of{\frac{a^2}{1 + a^2}}
  - \frac{1}{\sqrt{1 + a^2}} K \of{\frac{a^2}{1 + a^2}}.
\end{multline}

When $\hat{e}_1 ⊥ \hat{e}_2$, then we may write
$\hat{e}_1 · \vc{\hat{k}} = \cos{\left( ϕ_{\vK} - ϕ_1 \right)}$
and $\hat{e}_2 · \vc{\hat{k}} = \sin{\left( ϕ_{\vK} - ϕ_1 \right)}$,
thus
\begin{equation}
  \hat{e}_1 + i \hat{e}_2
  = \frac{v_0}{2} \sumK \frac{\abs{f_{\vK}}^2}{λ_{\vK}}
    \left[ \left( \hat{e}_1 + i \hat{e}_2 \right) · \vc{\hat{k}}
    \right] \vc{\hat{k}},
\end{equation}
and dotting both sides by $\hat{e}_1 - i \hat{e}_2$ gives
\begin{equation}
  2
  = \frac{v_0}{2} \sumK \frac{\abs{f_{\vK}}^2}
    {\sqrt{ξ_{\vK}^2 + \left( 1 / 2 \right) \abs{f_{\vK}}^2 \abs{χ_0}^2}}.
\end{equation}
and converting to integral form,
\begin{equation}
  1
  = \frac{π v_0}{2} ∫_{-ω}^{ω}
  \frac{\abs{f \of{ξ}}^2 \abs{ρ \of{ξ}} \dif ξ}
  {\sqrt{ξ^2 + \left( 1 / 2 \right) \abs{f \of{ξ}}^2 \abs{χ_0}^2}}.
\end{equation}
