\subsection{Optical BCS Theory in the Effective Model}
\label{s:appendix:bcs-optical}.

The general form of the full Hamiltonian which
includes a BCS term for the upper-valance band is
\begin{multline}
  \label{eq:hamiltonian:bcs:general}
  H = H^0 + H^V - μ N
  = \sumK ∑_{n, τ, σ}
    ξ_{τ σ}^n \ofK
    {c_{τ σ}^n}^† \ofK
    c_{τ σ}^n \ofK \\
    - \sumK \left[
    \cc{Δ}_{\vK}
    c_{γ'}^- \ofMK
    c_γ^- \ofK
    +
    Δ_{\vK}
    {c_γ^-}^† \ofK
    {c_{γ'}^-}^† \ofMK
    \right]
    + ε,
\end{multline}
where $γ = γ' = ±1$ or $γ = - γ' = 1$.

The optical transition rate can now be expressed
in terms of matrix elements between multiparticle states,
however, we must first determine the appropriate form
of the optical perturbation operator.

Earlier, we derived the approximate transition rate in terms
of the gradient of matrix elements of the Hamiltonian.
We now seek the corresponding form of the perturbation
written explicitly as an expansion in terms of Fock space operators.
We consider a perturbation of $H^0$ arising from minimal coupling;
we require it to match the result in the non-interacting limit.
First, we make the unitary transformation
$H^0_{\vK} + H^1_{\vK} = T_{\vK} H^0 T_{\vK}^{-1}$
where $H^0_{\vK}$ and $H^1_{\vK}$ are again defined by
\cref{appendix:optical:decomposition}.
As we only consider transitions for fixed $\vK$,
we may effectively set $H^1_{\vK} = 0$.
From the remaining expression,
\begin{equation}
  H^0_{\vK}
  = ∑_{ν, ν'} ∑_{τ, σ}
    H_{τ σ}^{ν ν'} \ofK
    T_{\vK}
    {a^ν_{τ σ}}^† \ofK
    a^{ν'}_{τ σ} \ofK
    T_{\vK}^{-1},
\end{equation}
the results for the noninteracting system
may be recovered through the substitution
\begin{equation}
  H_{τ σ}^{ν ν'} \ofK
  → H_{τ σ}^{ν ν'} \of{\vK + e \vc{A}}
  = H_{τ σ}^{ν ν'} \ofK + h_τ^{ν ν'},
\end{equation}
where in the dipole approximation,
$\vc{A} = 2 \re{\vc{ϵ} A_0 e^{- i ω t}}$,
$h_τ^{v v} = h_τ^{c c} = 0$,
and
\begin{equation}
  h_τ^{v c}
  = \cc{h}_τ^{c v}
  = 2 a t e A_0
    \left( τ \vc{\hat{x}} + i \vc{\hat{y}} \right)
    · \re{\left( \vc{ϵ} e^{- i ω t} \right)}.
\end{equation}
Thus we expect the optical perturbation operator to be
\begin{equation}
  H^A
  = \sumK ∑_{τ, σ}
    h_τ^{v c}
    {a^v_{τ σ}}^† \ofK
    a^c_{τ σ} \ofK
    + \hc
\end{equation}
Separating out the time dependence,
we may write this in the standard form
$H^A = H' e^{- i ω t} + H'^† e^{i ω t}$,
where
\begin{multline}
  H'
  = \sumK ∑_{τ, σ}
    H'_τ
    {a^v_{τ σ}}^† \ofK
    a^c_{τ σ} \ofK \\
  - \sumK ∑_{τ, σ}
    H'_{-τ}
    {a^c_{τ σ}}^† \ofK
    a^v_{τ σ} \ofK,
\end{multline}
and
\begin{equation}
  H'_τ
  = a t e A_0
    \left( τ \vc{\hat{x}} + i \vc{\hat{y}} \right)
    · \vc{ϵ}.
\end{equation}
Chancing basis, we may now recover the optical matrix element,
$P_{τ σ}^{n n'} \of{\vK, \vc{ϵ}}$,
now explicitly a function of the polarization vector,
\begin{multline}
  H^A
  = \sumK ∑_{τ, σ} ∑_{n, n'}
    \frac{e A_0}{m_0}
    P_{τ σ}^{n n'} \of{\vK, \vc{ϵ}}
    {c_{τ σ}^n}^† \ofK
    c_{τ σ}^{n'} \ofK.
\end{multline}
For circularly polarized light,
$\vc{ϵ_±} = \left( \vc{\hat{x}} ± i \vc{\hat{y}} \right) / \sqrt{2}$,
\begin{equation}
  P_{τ σ}^{+ -} \of{\vK, \vc{ϵ_±}}
  = ∓ τ \sqrt{2} a t m_0
    e^{± i ϕ}
    \sin^2 {\frac{\fnTheta{∓ τ}}{2}}.
\end{equation}
