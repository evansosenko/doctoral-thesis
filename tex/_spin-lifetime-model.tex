\section{Model}
\label{s:model}

\begin{figure}
  \centering
  \input{components/tikz-nonlocal_spin_valve/tex/_head}
  \begin{tikzpicture}[scale=1]
    \input{components/tikz-nonlocal_spin_valve/tex/_nonlocal_spin_valve}
  \end{tikzpicture}
  \caption{%
    The geometry of the nonlocal spin valve analyzed in this chapter is shown.
    There are two ferromagnetic electrodes placed on a conducting channel.
    Current $I$ flows into the left electrode,
    while the potential $V$ is measured at the right electrode.
    The nonlocal resistance is defined as the ratio $V / I$.
    For spin dependent phenomena, the relevant quantity of interest
    is the difference between the nonlocal resistance for the parallel
    and antiparallel orientations of magnetization of the two electrodes.
  }\label{fig:nonlocal_spin_valve}
\end{figure}

The assumed device geometry is shown in \cref{fig:nonlocal_spin_valve}.
Two ferromagnetic contacts ($F$) are deposited
on the normal semiconductor ($N$).
A spin-polarized current $I$ is injected through the contact at $x = 0$
and flows in the $x ≤ 0$ region of the semiconductor.
The voltage difference $V$ is measured at
$x = L$ between the contact and the semiconductor.
The nonlocal resistance is $\rNL = V / I$
\cite{PhysRevB.67.052409}.

Spin transport is modeled by identifying two spin channels
and their associated three-component spin electrochemical potentials $μ_{↑↓}$.
The majority channel is labeled as up,
while the minority channel is labeled as down.
The voltage difference is proportional to the spin accumulation
$μ_s = \left( μ_↑ - μ_↓ \right) / 2$ at $x = L$.
The spin accumulation in the semiconductor
is assumed to satisfy the steady-state Bloch diffusion equation
\begin{equation}
  D ∇^2 μ_s^N - \frac{μ_s^N}{τ} + ω × μ_s^N = 0.
\end{equation}
The key parameters are
the contact spacing $L$,
the diffusion constant $D$,
the spin lifetime $τ$,
the spin diffusion length $λ = \sqrt{D τ}$,
and $ω = \left( g μ_B / ℏ \right) B$ which is proportional to
the applied magnetic field $B$ and the gyromagnetic ratio $g = 2$.

For contacts which cover the width of the channel,
the transport is uniform along $y$.
Since the channel is two-dimensional, $μ_s^N$ will only vary along $x$.
We enforce the boundary condition $μ_s^N → 0$ at $x → ± ∞$
and the continuity of the current and spin current.
A detailed derivation is given in \cref{s:appendix} and reveals
\begin{equation}
  \label{eq:nonlocal_resistance}
  \rNL^± = ± p_1 p_2 R_N f.
\end{equation}
The overall sign corresponds to
parallel and antiparallel ferromagnetic alignments.
Specifically, we find a resistance scale
\begin{equation}
  R_N = \frac{λ}{W L} \frac{1}{σ^N} ,
\end{equation}
and the function
\begin{multline}
  \label{eq:f}
  f = \re \left( \left\{
        \vphantom{
          \frac{
            \sinh{ \left[ \left( L / λ \right) \sqrt{1 + i ω τ} \right] }
          }{\sqrt{1 + i ω τ}}
        }
        2 \left[ \sqrt{1 + i ω τ}
                 + \frac{λ}{2}
                   \left( \frac{1}{r_0} + \frac{1}{r_L} \right)
        \right]
        e^{\left( L / λ \right) \sqrt{1 + i ω τ}}
        \right. \right. \\ \left. {\left.
        + \frac{λ^2}{r_0 r_L} \frac{
            \sinh{ \left[ \left( L / λ \right) \sqrt{1 + i ω τ} \right] }
          }{\sqrt{1 + i ω τ}}
      \right\}}^{-1} \right).
\end{multline}

Note that $f$ is unitless and depends only on the scales
$L / λ$, $ω τ$, and $λ / r_i$.
The parameters $r_i$ with $i$ either $0$ for
the left contact or $L$ for the right are
\begin{equation}
  \label{eq:r-parameter}
  r_i = \frac{R_F + R_C^i}{\rSQ} W ,
\end{equation}
where $R_F$ is the resistance of the ferromagnet
and $R_C^i$ are the individual contact resistances,
$W$ is the graphene flake width, and
\begin{equation}
  \label{eq:square_resistance}
  \rSQ = W / σ^N ,
\end{equation}
is the graphene square (sheet) resistance
given in terms of the semiconductor conductivity $σ^N$.
The resistances $R_F$ and $R_C^i$ are the effective resistances
of a unit cross sectional area.
They are defined in \cref{eq:contact.resistance,eq:ferromagnet.resistance}.
To obtain an expression in terms of the ohmic resistances,
one must make the substitutions
$R_F → W_F W R_F$ and $R_C^i → W_F W R_C^i$,
where $W_F$ is the contact width, i.e., $W_F W$ is the contact area.
We will use the same symbols for either resistance type
when the meaning is clear.
The polarizations $p_1$ and $p_2$, defined in \cref{eq:polarizations},
model the effective current injection.
They depend on the resistances and the spin polarizations
of the semiconductor and the individual contacts.

The expression $Δ \rNL = \abs{\rNL^+ - \rNL^-}$
measures the difference in signal between
parallel and antiparallel field alignments.
We combine $P^2 = \abs{p_1 p_2}$
\footnote{
  Assuming the polarizations $P_σ^F$ and $P_Σ^L$
  have the same sign bounds $P ≤ 1$.
},
and write
\begin{equation}
  \label{eq:nonlocal_resistance.difference}
  Δ \rNL = 2 P^2 R_N \abs{f},
\end{equation}
with
\begin{equation}
  R_N = \frac{λ}{W} \frac{1}{σ_G},
\end{equation}
where $σ_G = σ^N L$ is the graphene conductance normally given in units of
$\si{\milli\siemens} = {\left( \si{\milli\ohm} \right)}^{-1}$.
