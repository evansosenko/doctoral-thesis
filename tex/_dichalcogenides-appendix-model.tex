\section{Dichalcogenide System}
\label{appendix:model}

\subsection{Model}

The minimal model
\cite{PhysRevLett.108.196802}
for single-layer TMDCs in the low energy bands
about the $± \vc{K}$ points is a two-band tight binding model with bases
\begin{subequations}
\label{eq:orbitals}
  \begin{alignat}{2}
    & \ketOrbital{\text{c}} && = \Ket{d_{z^2}} ⊗ \Ket{σ}, \\
    & \ketOrbital{\text{v}} && = \frac{1}{\sqrt{2}}
        \left( \Ket{d_{x^2 - y^2}} + i τ \Ket{d_{xy}} \right) ⊗ \Ket{σ}.
  \end{alignat}
\end{subequations}
Here, $\Ket{σ}$ is the spin state,
and $\Ket{d_{xy}}$ and $\Ket{d_{x^2 - y^2}}$
refer to the angular momentum orbitals
in the symmetry group $E \left( d_{xy}, d_{x^2 - y^2} \right)$.
The valley index $τ = ±1$, corresponding to the $± \vc{K}$ points,
and the spin index $σ = ±1$ (or $σ = ↑,↓$), corresponding to the $z$-component
of the spin through $s_z = σ / 2$, are good quantum numbers.
The periodic Bloch basis states
(see \cref{s:appendix}) are
\begin{equation}
  \ketOrbitalK{ν}
  = \frac{T_{\vK +  τ \vc{K}}}{\sqrt{N}}
    ∑_{n = 1}^N e^{i \left( \vK + τ \vc{K} \right) ⋅ \vRn{n}}
    T \of{\vRn{n}} \ketOrbital{ν},
\end{equation}
where $ν = \text{c}, \text{v}$ is the orbital basis index
and $N$ is the number of \ce{M}-type atoms in the system.

The effective Bloch Hamiltonian is
\begin{multline}
  \label{eq:hamiltonian:effective}
  H_τ^0 \ofK
  = a t \left(τ k_x σ_x + k_y σ_y \right) ⊗ I_2 \\
    + \frac{Δ}{2} σ_z ⊗ I_2 - λ τ \left(σ_z - 1 \right) ⊗ S_z,
\end{multline}
where $Δ$ is the energy gap, $2 λ$ is the spin splitting in the lower band,
$a$ is the lattice constant, and $t$ is the effective hopping integral.
Note that $Δ > 2 λ$.
The operators $σ_i$ are Pauli operators acting on the orbital space
such that $σ_z \ketOrbitalK{\text{c}} = \ketOrbitalK{\text{c}}$
and $σ_z \ketOrbitalK{\text{v}} = - \ketOrbitalK{\text{v}}$.
\Cref{eq:hamiltonian:effective}
can be written in matrix form in the Bloch orbital basis,
\begin{equation}
  \left[ H_{τ σ}^0 \ofK \right] =
    \left[
    \begin{matrix}
      \dfrac{Δ}{2}                     & a t \left( τ k_x - i k_y \right) \\
      a t \left( τ k_x + i k_y \right) & λ τ σ - \dfrac{Δ}{2}
    \end{matrix}
    \right].
\end{equation}

\subsubsection{Energy}

The Hamiltonian is diagonal in all but the binary orbital index,
thus the eigenstates may be written as elements of a Bloch sphere,
\begin{equation}
  \begin{aligned}
    \ketEigenstateKPhi{n} =
                   & \cos{\frac{\fnTheta{n}}{2}} \ketOrbitalKPhi{c} \\
      + e^{-i τ ϕ} & \sin{\frac{\fnTheta{n}}{2}} \ketOrbitalKPhi{v},
  \end{aligned}
\end{equation}
where $k_x + i τ k_y = k e^{i τ ϕ}$,
$k = \abs{\vK}$, and
\begin{equation}
  \tan{\frac{\fnTheta{n}}{2}}
  = \frac{a t τ k}{\dfrac{Δ}{2} - \fnEnergy{-n} \of{k}}
  = \frac{a t τ k}{\fnEnergy{n} \of{k} - \fnEnergy{-} \of{0}}.
\end{equation}
Note the relations
$θ_{-↓}^{n} \of{k} = 2π - θ_{+↑}^{n} \of{k}$,
$\fnTheta{+} = \fnTheta{-} - τ π$,
and $ϕ_{-\vK} = π + ϕ_{\vK}$.
We will primarily write equations in terms of
$θ_{\vK} = θ_{+↑}^- \of{\abs{\vK}}$.
The energies are given by
\begin{equation}
  \fnEnergy{n} \of{k}
  = \frac{1}{2} \left( λ τ σ + n \sqrt{{(2 a t k)}^2
    + {\left( Δ - λ τ σ \right)}^2} \right).
\end{equation}
For a fixed band, we have the inverse relation,
\begin{equation}
  {\left( \frac{a t k}{Δ / 2} \right)}^2
  = {\left( \frac{2 E}{Δ} \right)}^2
    + 2 τ σ \left( \frac{λ}{Δ} \right) \left( 1 - \frac{2 E}{Δ} \right) - 1,
\end{equation}
where $E > Δ / 2$ for $n = 1$ and
$E < - \left( Δ / 2 - λ τ σ \right)$ for $n = -1$.

% TODO Add band structure figure.

We adopt the following notation.
The Fock fermion operators corresponding to
the momentum-space orbitals and eigenstates are defined by
\begin{subequations}
  \begin{align}
    {a^ν_{τ σ}}^† \ofK \Ket{0} & = \ketOrbitalTB{ν}, \\
    {c^n_{τ σ}}^† \ofK \Ket{0} & = \ketEigenstateTB{n}.
  \end{align}
\end{subequations}
The coefficients $M_{τ σ}^{ν n} = \cc{W}_{τ σ}^{ν n}$
relating the two basis are
\begin{subequations}
  \begin{align}
    \ketEigenstateK{n} & = ∑_ν M_{τ σ}^{ν n} \ofK \ketOrbitalK{ν}, \\
    \ketOrbitalK{ν}    & = ∑_n W_{τ σ}^{n ν} \ofK \ketEigenstateK{n}.
  \end{align}
\end{subequations}
The same relation holds between $\Ket{ϕ_{ν \vK}}$ and $\Ket{ψ_{n \vK}}$.

\subsubsection{Optical Transitions}

For the noninteracting model, the relevant matrix element for
the single particle optical transition rate
in the dipole approximation is given by
\cref{appendix:optical:approx},
\begin{equation}
  \vc{P}_{τσ}^{n n'} \ofK
  ⋍ m_0 ∑_{ν, ν'} \cc{M}_{τ σ}^{n ν} M_{τ σ}^{ν' n'}
    ∇_{\vK} \left( {\left[ H_{τσ}^0 \ofK \right]}_{ν ν'} \right).
\end{equation}
One can verify the relation
\begin{equation}
  P_{τ σ}^{n n'} \of{\vK, \vc{ϵ}}
  = \vc{ϵ} · \vc{P}_{τσ}^{n n'} \ofK
\end{equation}
which summarizes the equivalence of this computation
and the one given in
\cref{s:appendix:bcs-optical}.

A brief note about the units of $P$ above.
To conveniently  express $P$ in units of energy,
multiply by $c / \si{\planckbar}$.
Typically, $a t$ is given in \si{\angstrom\electronvolt},
the electron mass is given in units
of $\si{\mega \electronvolt} / c^2$,
\si{\planckbar} in \si{\electronvolt\second},
and $c$ in \si{\angstrom\per\second}.
Thus when written as $(c / ℏ) P = p E_P$,
where $p$ is a unitless function of the energy,
the important overall energy scale is
$E_P = at m_0 / (c ℏ) × 10^3 \si{\giga\electronvolt}$,
where the symbols in $E_p$ are the numerical magnitudes of the
quantities when expressed in the assumed units above.
In particular,
$E_P = at · \SI{0.259}{\giga\electronvolt}$,
and for $at = 3.2$, $E_p = \SI{0.83}{\giga\electronvolt}$.
